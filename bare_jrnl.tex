%\documentclass[10pt,journal,compsoc]{IEEEtran}
\documentclass[lettersize,journal]{IEEEtran}
%\IEEEoverridecommandlockouts
%\usepackage{geometry} \geometry{letterpaper,total={6.5in,9in}}
%\usepackage{usenix,epsfig,endnotes}
%\usepackage{titling}
%\usepackage{flushend}
%\usepackage[small,compact]{titlesec}
%\usepackage[pagebackref]{hyperref}
%\usepackage{fancyhdr}
%\usepackage{mdframed, fancyvrb}
%\usepackage[scaled]{beramono}
%\usepackage{times}
%\usepackage{titlesec}
\usepackage{afterpage}
\usepackage{xspace}
\usepackage{multirow}
\usepackage{lastpage}
\usepackage{comment}
\usepackage{fancybox, fancyvrb, calc}
%\usepackage{subfigure}
\usepackage{subfig}
\usepackage[inline]{enumitem}
\usepackage{amsmath, amssymb}
%\usepackage[bf]{caption}
\usepackage{cite,url}
%\usepackage[breaklinks]{hyperref}
\usepackage{breakurl}
%\usepackage[usenames, dvipsnames, svgnames, table]{xcolor}
\usepackage{graphicx}
\usepackage{pbox}
\usepackage{algorithm, algorithmicx,algpseudocode}
\usepackage[amsmath,thmmarks,framed,hyperref]{ntheorem}
\usepackage{balance}
\usepackage{wrapfig}
\usepackage{tikz, pgfplots}
\usepgfplotslibrary{groupplots} 
\usetikzlibrary{patterns,matrix,positioning,shadows,backgrounds,arrows,calc,fit,automata,shapes.geometric,arrows,decorations.pathreplacing,decorations.markings}
\usepackage[T1]{fontenc}
\usepackage[utf8]{inputenc}
\usepackage{mathptmx}
% \usepackage[charter]{mathdesign}
% \usepackage[mathcal]{euscript}
%\usepackage{helvetica}
\usepackage{fullpage}
\usepackage{enumitem}
\usepackage{sidecap}
\usepackage{hyperref}
\usepackage{caption} 
\usepackage{epstopdf}
\usetikzlibrary{arrows}
%\captionsetup[table]{skip=-4pt}
\Urlmuskip=0mu plus 1mu
%\hypersetup{
  %colorlinks=true,      % false: boxed links; true: colored links
  %linkcolor=blue,       % color of internal links
  %citecolor=magenta,    % color of links to bibliography
  %filecolor=cyan,       % color of file links
  %urlcolor=red          % color of external links
%}

\usepackage{etoolbox}

\usepackage{booktabs}
\usepackage{enumitem}
\usepackage{xspace}
\usepackage{numprint, hyperref}
\usepackage[capitalise]{cleveref}
\usepackage{xurl}

\makeatletter
\patchcmd{\ttlh@hang}{\parindent\z@}{\parindent\z@\leavevmode}{}{}
\patchcmd{\ttlh@hang}{\noindent}{}{}{}
\makeatother



% definitions
\def\ie{{i.e.}}
\def\eg{{\em e.g.}\xspace}
\def\tt{\texttt}
%\def\bf{\textbf}
%\def\name{I/O-TT\xspace}
%\def\name{Tasador\xspace}
\def\name{TASADOR\xspace}
%\def\np{\numprint}
\def\kw{\textcolor{orange}}

% BEGIN Theorem environment settings
\theoremstyle{plain}
\theorempreskipamount2pt
\theorempostskipamount2pt
\newtheorem{myTheorem}{Theorem}[section]
\newtheorem{myLemma}[myTheorem]{Lemma}
\newtheorem{myCorollary}[myTheorem]{Corollary}
\newtheorem{myClaim}{Claim}[section]
\newtheorem{myRemark}{Remark}
\newtheorem{myExample}{Example}[section]
\newtheorem{myConjecture}{Conjecture}[section]
\newtheorem{myDef}{Definition}[section]

\theorembodyfont{}
\newtheorem*{myNote}{Note}

\definecolor{boxcolor}{gray}{0.9}
\newenvironment{colframe}{%
  \begin{Sbox}
    \begin{minipage}
      {0.96\columnwidth}
    }{%
    \end{minipage}
  \end{Sbox}
  \begin{center}
    \colorbox{boxcolor}{\TheSbox}
  \end{center}
}


% Proofs get merged into text, with a "Proof" flag and box at the end
\theoremstyle{nonumberplain}
\theoremheaderfont{\normalfont\bfseries}
\newcommand{\proofsubhead}[1]{\noindent{\normalfont\bfseries\boldmath#1}}
\theoremsymbol{$\blacksquare$} % {~{\rule[0.2ex]{1.17ex}{1.2ex}}}
\theorembodyfont{}
\theorempreskipamount0pt
\theorempostskipamount0pt
\theoremindent0pt\relax

\newtheorem{myProof}{Proof}
\newtheorem{myProofSketch}{Proof Sketch}

\theoremstyle{break}
\theoremheaderfont{\bfseries}
\theorembodyfont{}
\theoremsymbol{}
% END Theorem environment settings

\newenvironment{denseitemize}{
\begin{itemize}[topsep=2pt, partopsep=0pt, leftmargin=1.5em]
  \setlength{\itemsep}{4pt}
  \setlength{\parskip}{0pt}
  \setlength{\parsep}{0pt}
}{\end{itemize}}

\newenvironment{denseenum}{
\begin{enumerate}[topsep=2pt, partopsep=0pt, leftmargin=1.5em]
  \setlength{\itemsep}{4pt}
  \setlength{\parskip}{0pt}
  \setlength{\parsep}{0pt}
}{\end{enumerate}}

\definecolor{oldcolor}{HTML}{c66541}
\newcommand{\old}[1]{{\color{oldcolor}{#1}}}
\newcommand{\rc}[1]{{\color{red}{#1}}}
\newcommand{\jaehyun}[1]{{\color{blue}{#1}}}

\newcommand{\cut}[1]{}
\renewcommand{\ttdefault}{txtt}
\newcommand{\paragraphb}[1]{\vspace{0.075in}\noindent{\bf #1.}}
\newcommand{\paragrapha}[1]{\vspace{0.075in}\noindent{\bf #1}}
\newcommand{\paragraphc}[1]{\vspace{0.075in}\noindent{#1}}
\newcommand{\todo}[1]{\textcolor{Red}{\{#1\}}}

%\titlespacing*{\section}{0pt}{1em}{0.5em}
%\titlespacing*{\subsection}{0pt}{1em}{0.5em}
%\titlespacing*{\subsubsection}{0pt}{1em}{0.5em}
% to be able to draw some self-contained figs
\usepackage{graphicx}
\usepackage{epstopdf}
% correct bad hyphenation here
\hyphenation{op-tical net-works semi-conduc-tor}
\def\BibTeX{{\rm B\kern-.05em{\sc i\kern-.025em b}\kern-.08em
    T\kern-.1667em\lower.7ex\hbox{E}\kern-.125emX}}

\begin{document}
\bstctlcite{IEEEexample:BSTcontrol}
%
% paper title
% Titles are generally capitalized except for words such as a, an, and, as,
% at, but, by, for, in, nor, of, on, or, the, to and up, which are usually
% not capitalized unless they are the first or last word of the title.
% Linebreaks \\ can be used within to get better formatting as desired.
% Do not put math or special symbols in the title.
\title{TASADOR: A Machine Learning Framework for Network Bandwidth to CPU Translations}
%
%
% author names and IEEE memberships
% note positions of commas and nonbreaking spaces ( ~ ) LaTeX will not break
% a structure at a ~ so this keeps an author's name from being broken across
% two lines.
% use \thanks{} to gain access to the first footnote area
% a separate \thanks must be used for each paragraph as LaTeX2e's \thanks
% was not built to handle multiple paragraphs
%

%\author{Michael~Shell,~\IEEEmembership{Member,~IEEE,}
%        and~Jane~Doe,~\IEEEmembership{Life~Fellow,~IEEE}% <-this % stops a space
%\thanks{M. Shell was with the Department
%of Electrical and Computer Engineering, Georgia Institute of Technology, Atlanta,
%GA, 30332 USA e-mail: (see http://www.michaelshell.org/contact.html).}% <-this % stops a space
%\thanks{J. Doe and J. Doe are with Anonymous University.}% <-this % stops a space
%\thanks{Manuscript received April 19, 2005; revised August 26, 2015.}}
%\author{Kyungwoon Lee, Sooyeon Woo, Jaehyun Hwang, ChuckYoo,~\IEEEmembership{Member,~IEEE,}}%


% note the % following the last \IEEEmembership and also \thanks - 
% these prevent an unwanted space from occurring between the last author name
% and the end of the author line. i.e., if you had this:
% 
% \author{....lastname \thanks{...} \thanks{...} }
%                     ^------------^------------^----Do not want these spaces!
%
% a space would be appended to the last name and could cause every name on that
% line to be shifted left slightly. This is one of those "LaTeX things". For
% instance, "\textbf{A} \textbf{B}" will typeset as "A B" not "AB". To get
% "AB" then you have to do: "\textbf{A}\textbf{B}"
% \thanks is no different in this regard, so shield the last } of each \thanks
% that ends a line with a % and do not let a space in before the next \thanks.
% Spaces after \IEEEmembership other than the last one are OK (and needed) as
% you are supposed to have spaces between the names. For what it is worth,
% this is a minor point as most people would not even notice if the said evil
% space somehow managed to creep in.



% The paper headers
%\markboth{Journal of \LaTeX\ Class Files,~Vol.~14, No.~8, August~2015}%
%{Shell \MakeLowercase{\textit{et al.}}: Bare Demo of IEEEtran.cls for IEEE Journals}
% The only time the second header will appear is for the odd numbered pages
% after the title page when using the twoside option.
% 
% *** Note that you probably will NOT want to include the author's ***
% *** name in the headers of peer review papers.                   ***
% You can use \ifCLASSOPTIONpeerreview for conditional compilation here if
% you desire.


% If you want to put a publisher's ID mark on the page you can do it like
% this:
%\IEEEpubid{0000--0000/00\$00.00~\copyright~2015 IEEE}
% Remember, if you use this you must call \IEEEpubidadjcol in the second
% column for its text to clear the IEEEpubid mark.


% As a general rule, do not put math, special symbols or citations
% in the abstract or keywords.
%\IEEEtitleabstractindextext{%
\maketitle
\begin{abstract}
In virtualized cloud environments, meeting the network bandwidth requirements of virtual machines (VMs) is challenging due to the lack of accurate methods for translating target network bandwidth into precise CPU allocations.
This translation is non-trivial, as outcomes vary depending on factors such as workloads, message sizes, and hardware configurations.
For instance, our experiments reveal that for the same CPU allocation, network bandwidth can vary between $0.4\times$ and $3.5\times$.
In this paper, we introduce \name, a machine learning (ML) framework for network bandwidth to CPU translations. \name predicts ``appropriate'' CPU budgets for new workloads to meet their bandwidth requirements.
We demonstrate that \name completes data collection and training in ${\sim}16$ minutes while achieving target bandwidth accurately with ${\sim}6.5\times$ less CPU utilization compared to existing schemes. Our experiments also confirm that \name is effective across diverse workloads, message sizes, and hardware configurations.
\end{abstract}



% Note that keywords are not normally used for peerreview papers.
\begin{IEEEkeywords}
Cloud computing, virtual machines, network bandwidth, performance requirements.
\end{IEEEkeywords}



% For peer review papers, you can put extra information on the cover
% page as needed:
% \ifCLASSOPTIONpeerreview
% \begin{center} \bfseries EDICS Category: 3-BBND \end{center}
% \fi
%
% For peerreview papers, this IEEEtran command inserts a page break and
% creates the second title. It will be ignored for other modes.

%\IEEEpeerreviewmaketitle
\maketitle
%\IEEEdisplaynontitleabstractindextext

\section{Introduction}
\IEEEPARstart{R}{cent} cloud-based applications and online services increasingly require network service level objectives (SLOs) in terms of network bandwidth in order to ensure a targeted service quality~\cite{qiu2020firm,kannan2019grandslam,jang2015silo,sriraman2018mutune,jyothi2016morpheus,kumar2019picnic}.
As an example, a webserver that caters to $1,000,000$ monthly visitors, with an average of four-page views per user and an average data transfer of $10$MB per page, would necessitate an average network bandwidth of around $120$Mbps~\cite{loadninjahowto,abdelzaher2002performance, romanohowto}.
In particular, web servers often deliver HTTP-based video streaming services, where video quality is determined by the measured network bandwidth. Thus, insufficient bandwidth can lead to poor quality of service or significant playback delays.
Another major use case for network bandwidth requirements would be network functions virtualization~\cite{chintapalli2023ravin,tootoonchian2018resq,yen2020meeting, han2024byways, yoon2024mmtls, xu2024cyberstar}
%which must prevent performance interference between co-located virtual machines (VMs).
, which similarly involves processing large volumes of network packets at high rates, with performance requirements varying depending on specific network functions (NFs)~\cite{xu2024cyberstar}. It is critical for NFs to meet their target throughput (bandwidth), especially when multiple NFs are deployed across different VMs on the same physical server.

Given that VMs are a key element in cloud computing environments~\cite{chen2020alita,suo2018ananalysis}, the ability to meet the bandwidth requirements of VMs has emerged as a critical determinant of service quality within cloud computing~\cite{jang2015silo,kumar2019picnic}.
In addition, for user-facing microservice applications, latency is often a key determinant of service quality~\cite{park2021graf}. Recent studies~\cite{wang2024autothrottle,zhang2021sinan,qiu2020firm,park2021graf} have therefore focused on meeting network latency requirements by dynamically adjusting resource allocations for containers. 
Since these containers are typically deployed on VMs \kw{(including microVMs such as Firecracker of Amazon AWS)} rather than bare-metal servers~\cite{cai2023automan,wang2024autothrottle,entrialgo2024joint,bermejo2024goodness}, it is essential for VMs to be allocated sufficient CPU resources and network bandwidth to meet the network latency requirements.

\kw{Prior work has primarily pursued two directions—network scheduling and vCPU prioritization—to meet the bandwidth requirements. However, neither approach fully addresses meeting network bandwidth requirements in VMs. First, network scheduling~\cite{jang2015silo,jeyakumar2013eyeq,russell2008virtio} controls per-VM transmission rates via queueing and shaping, yet it implicitly assumes that the VM will receive \textbf{enough CPU time} to process packets at the desired rate; when CPU becomes the bottleneck (e.g., high packet-per-second traffic or CPU contention), the CPU scheduler—being oblivious to bandwidth requirements—can cap packet processing and prevent the target bandwidth from being reached even if rate limits are configured. 
Second, vCPU prioritization (including weight-based CPU sharing)~\cite{jia2018effectively,jia2020vsmt,xu2013vturbo,suo2017preserving} attempts to favor VMs with bandwidth requirements, but it does not directly answer the central question of \textit{how much CPU is needed for a given bandwidth target}. Without an accurate mapping, VMs have difficulty in meeting the bandwidth requirements accurately and efficiently.}

\kw{These limitations highlight three challenges in meeting network bandwidth SLOs for VMs. First, the CPU needed to sustain a target bandwidth is highly workload-, message-size-, and hardware-dependent, so a fixed Mbps→CPU rule can easily over- or under-provision. Second, bandwidth delivery depends on a coupled host–guest packet-processing pipeline (guest stack/virtio and host vhost/qemu/kernel), where the bottleneck can shift with contention, making naïve provisioning unstable. Third, real services are often multi-threaded and run in multi-vCPU VMs, introducing scaling and scheduling effects that complicate both accurate modeling and robust CPU enforcement for bandwidth requirements.}

\kw{This paper presents the design and implementation of \name, a general framework for translating network bandwidth SLOs into CPU allocations for VMs. \name tackles the three challenges above as follows. First, to handle the strong dependence of CPU demand on workload behavior, message size, and hardware, \name performs lightweight per-setting profiling: for each target workload specification (including message size and platform), it collects training samples by sweeping CPU allocations and measuring the resulting bandwidth, requiring up to 16 minutes in total for data collection and model building. Second, to account for the coupled host–guest packet-processing pipeline, \name explicitly separates their effects by learning two regression models, one for guest-side CPU usage and one for host-side CPU usage, and composes their predictions to estimate the CPU needed to sustain the target bandwidth. Third, to remain robust to multi-threaded services and varying numbers of vCPU, \name enforces allocations only at the host level (rather than controlling in-guest CPU usage). This enables \name to be independent of the guest’s internal threading model and vCPU configuration while still ensuring bandwidth requirements.}

\kw{More broadly, \name leverages ML to provide a fast and deployable bandwidth-to-CPU translation, but it differs from prior ML-based resource managers~\cite{qiu2020firm,zhang2021sinan,qiu2022simppo} that jointly optimize multiple resources (e.g., CPU, cache, memory, and disk) and therefore typically require higher-dimensional feature sets and longer training cycles. By narrowing the objective to bandwidth-to-CPU translation, \name can rely on a lightweight regression model that can be trained quickly for each new workload specification. Specifically, \name operates in three phases (\S\ref{subsec:framework}): (i) training data collection, where \name runs the target workload under a sweep of CPU allocations and records the achieved bandwidth to capture the bandwidth–CPU relationship; (ii) model training and CPU prediction, where \name trains a regression model from the collected data (we evaluate Linear Regression, Support Vector Regression, and Random Forest Regression, and select Random Forest Regression based on prediction accuracy in \S\ref{subsec:model}); and (iii) host-level CPU enforcement, where \name uses the predicted CPU requirement for the target bandwidth and applies the corresponding host-side CPU control to the VM.
Our evaluation shows that \name satisfies network bandwidth requirements consistently across diverse workloads, message sizes, and hardware configurations. Moreover, \name improves CPU allocation efficiency by up to 6.5$\times$ and achieves more stable bandwidth delivery—up to 3.0$\times$ lower bandwidth variation than existing schemes—for representative applications such as Webserver and Memcached.}

\kw{Compared to prior work, \name (1) formulates \emph{bandwidth-to-CPU translation} as the primary goal for VMs,
(2) decomposes prediction into \emph{guest and host CPU effects} to capture virtualization-specific packet-processing costs,
and (3) enforces only \emph{host-side CPU quotas} to remain robust to guest vCPU and threading configurations.
}

\kw{Our contributions are as follows:}

\begin{itemize}
    \item We design and implement \name, a general framework that translates per-VM bandwidth requirements into host CPU allocations.
    \item \name builds a workload- and environment-specific ML model with lightweight profiling, completing data collection and training in ${\sim}16$ minutes.
    \item \name models guest and host CPU effects separately, but enforces allocations only at the host to remain independent to the number of vCPUs.
    \item Across representative applications and traffic settings, \name improves CPU allocation efficiency by up to 6.5$\times$ and reduces bandwidth variation by up to 3.0$\times$ compared to existing schemes.
\end{itemize}

\begin{comment}
This paper presents the design and implementation of \name, a general framework for network bandwidth to CPU translations.
To cope with a variety of VM workloads and hardware configurations, \name leverages state-of-the-art machine learning (ML) techniques. Our key insight is that, unlike recent ML schemes~\cite{qiu2020firm,zhang2021sinan, qiu2022simppo}, which consider multiple resources, including cache, memory, and disk for resource allocation, focusing on CPU translation allows for the use of a simpler ML model. This enables \name to build a well-trained model quickly for diverse target workloads. When a new workload specification is received, \name operates in three main phases (\S\ref{subsec:framework}): (1) training data collection, (2) model training and CPU prediction, and (3) CPU allocation enforcement. During training data collection, \name measures the network bandwidth of the target workload under varying CPU allocations to identify the correlation between network bandwidth and CPU allocation.
Next, \name trains its ML model; after evaluating three representative regression models---Linear Regression, Support Vector Regression, and Random Forest Regression---we select Random Forest Regression due to its superior prediction accuracy (\S\ref{subsec:model}).
Lastly, \name predicts the appropriate CPU allocation required to meet the target network bandwidth using the trained model and enforces this allocation on the corresponding VMs (\S\ref{sec:eval}).
Note that \name targets VM-level CPU allocation for meeting network bandwidth requirements, which is different from recent studies for container-level CPU allocation for micro-services~\cite{qiu2020firm,cai2023automan,wang2024autothrottle}. As the containers run on VMs in cloud data centers, our work is complementary to the previous studies to offer proper CPU for both VMs and containers.

\name provides several key benefits as follows.
First, \name requires only ${\sim}16$ minutes to collect a dataset and train its ML model for new workloads; this enables quick adaptation to various workloads and server environments.
Second, \name requires no modifications inside VMs; this allows existing applications and guest OSes to operate without any modifications. This greatly helps the deployability of \name.
Third, \name improves CPU allocation efficiency by ${\sim}6.5\times$ compared to existing schemes. 
Finally, by collaborating with the work-conserving Linux CPU scheduler, \name dynamically redistributes underutilized CPU resources to other VMs, leading to higher overall system utilization.
We evaluate \name across diverse workloads and configurations, varying network bandwidth requirements, message sizes, and the number of concurrent requests. Our results demonstrate that \name achieves target network bandwidth more accurately, with ${\sim}3.0\times$ lower bandwidth variation compared to existing schemes, for widely used datacenter applications such as Webserver and Memcached 
\end{comment}
%\section{Background and motivation}
\vspace{-0.1in}
\section{Understanding Network Bandwidth to CPU Translations}
%\section{Understanding CPU Translations}
\label{sec:motivation}

This section presents a deep dive into why the network bandwidth to CPU translation is crucial for achieving the target bandwidth. We carry out experiments with two existing schemes---network scheduling and vCPU prioritization---and measure how inaccurate they are.
We first describe our setup (\S\ref{subsec:config}), and then discuss several results and insights of our benchmarks with the existing schemes (\S\ref{subsec:existing}). We then extend our insights to other workloads and hardware configurations (\S\ref{subsec:different}).
\vspace{-0.2in}
\subsection{Measurement Setup}
\label{subsec:config}
We use a testbed with two physical machines, each with a $10$Gbps NIC, connected to each other via a $10$Gbps network switch.
Also, we use KVM~\cite{kivity2007kvm}, a popular hypervisor integrated into the Linux kernel, and create a single VM running on Linux kernel 5.4, allocating one vCPU to the VM.
This section focuses on a single-core performance to better understand per-core network bandwidth to CPU translation (see \S\ref{sec:eval} for multicore evaluation). To this end, we pin the VM and the Host processes\footnote{The Host process represents a hypervisor thread in KVM that handles I/O device emulation.} on a single physical core.
Details of system configurations are described in \S\ref{sec:eval}.

%\renewcommand{\arraystretch}{1.1}
\begin{table}
\footnotesize
\centering
\resizebox{\linewidth}{!}{
\begin{tabular}{|c|c|c|} \hline
~~Configuration~~  & CPU (Intel Xeon)  & ~~~Memory~~~  \\ \hline\hline
Config1			& E5-2650v3@2.6GHz with $10$ cores & $256$GB \\ \hline
Config2			& E5-2650v2@2.6GHz with $8$ cores & $64$GB \\ \hline
Config3			& Silver 4310@2.1GHz with $24$ cores & $128$GB \\ \hline
\end{tabular}}
\caption{Hardware configurations.}
\label{table:configs}
\vspace{-0.1in}
\end{table}
 
For the network scheduling scheme, we use Linux traffic control ({\tt tc})~\cite{hubert2002linux}, a widely used network scheduling technique in recent studies~\cite{jeyakumar2013eyeq,qiu2020firm,autothrottle23}.
The vCPU prioritization scheme prioritizes the vCPU of the SLO-configured VMs (\ie, those with bandwidth requirements) over non-SLO VMs~\cite{jia2020vsmt,suo2017preserving,jia2018effectively}. Unfortunately, to the best of our knowledge, there is no publicly available implementation for testing this approach, so in our paper, we implement the vCPU prioritization by varying the ``vCPU share'' of the VM~\cite{bouron2018battle}, where higher vCPU share indicates a higher priority in CPU scheduling.
Here, we define four prioritization levels: {\tt default}, {\tt prio1}, {\tt prio2}, and {\tt prio3}, corresponding to vCPU share values of $1024$, $2048$, $4096$, and $8192$, respectively.
We then measure the network bandwidth and CPU utilization of the VM (Guest) and hypervisor (Host) for both schemes. To achieve this, we employ Netperf benchmark, which continuously transmits a size of messages. 
Lastly, to draw comparisons across various workloads and hardware configurations, we use Webserver and Memcached workloads, along with the hardware configurations listed in Table~\ref{table:configs}; unless
otherwise stated, {\tt Config1} is used as the default configuration.
\vspace{-0.2in}
\begin{figure}
\centering
  \subfloat[Message size: $64$B.] {
    \includegraphics[width=0.22\textwidth]{figure/tc_64_fin.pdf}
    \label{fig:network_perf1}
  }
  \subfloat[Message size: $1024$B.] {
    \includegraphics[width=0.22\textwidth]{figure/tc_1024_fin.pdf}
    \label{fig:network_perf2}
  }
%\vspace{-0.1in}
\caption{Network scheduling scheme ({\tt tc}) requires sufficient CPU resources for the Guest to meet the network bandwidth requirement. %See \S\ref{subsec:existing} for more details.
}
\label{fig:network_perf}
\vspace{-0.2in}
\end{figure}
%\vspace{-0.1in}
\subsection{Results of Two Schemes: Imperfect Translations}
\label{subsec:existing}
We begin our analysis by evaluating Netperf benchmarks with varying message sizes.

\paragraphb{Network scheduling}
In Figure~\ref{fig:network_perf}, we vary the target bandwidth (SLO) from $100$Mbps to $400$Mbps and apply rate-limiting to the VM using {\tt tc}.
While {\tt tc} is generally effective at controlling the transmission rate through packet scheduling, Figure~\ref{fig:network_perf}(a) reveals its inability to meet even $200$Mbps SLO when the message size is small ($64$B). 
This limitation stems primarily from the Linux CPU scheduler, Completely Fair Scheduling (CFS), which allocates a fair share of CPU time to both the Guest and the Host processes, %(see the green and blue bars in Figure~\ref{fig:network_perf}(a)),
However, the Guest process requires more CPU time than its fair share in order to generate more network traffic to meet the bandwidth requirement, a factor that CFS is unaware of. As a result, its network bandwidth reaches only ${\sim}150$Mbps under the given CPU allocation, even when the bandwidth requirement is higher.
Using a larger message size (\eg, $1024$B) reduces the per-message processing overhead~\cite{cai2021} within the Guest, saving CPU cycles.
This enables the Guest to meet network bandwidth requirements of up to $200$Mbps, as shown in Figure~\ref{fig:network_perf}(b). 
However, it still fails to achieve higher bandwidth requirements when the CPU time allocated by CFS becomes the bottleneck again.
The results mean imperfect translations due to SLO-unaware CPU scheduling.

\paragraphb{vCPU prioritization}
%To investigate whether allocating more CPU time to the Guest can improve network bandwidth, we introduce vCPU prioritization, allowing more CPU resources to be allocated to the Guest beyond its fair share. 
\kw{We implement “vCPU prioritization” as a cgroup weight adjustment by increasing \tt{cpu.shares} for the cgroup that contains the VM’s vCPU threads. In Linux CPU scheduling\footnote{This includes Linux’s CFS in earlier kernels and the EEVDF-based scheduler in newer kernels.}, CPU time is apportioned across runnable cgroups roughly in proportion to their weights; therefore, a higher \tt{cpu.shares} increases the VM’s relative CPU share under contention. However, it does not guarantee additional CPU time; it only increases the VM’s relative scheduling share when the vCPU threads are runnable and the host is contended.}
Figure~\ref{fig:vcpu_perf}(a) shows that the network bandwidth can increase up to $454$Mbps with the {\tt prio3} priority when the message size is $64$B.
The problem is that vCPU prioritization often leads to CPU over-provisioning due to its coarse-grained CPU allocation.
For instance, in Figure~\ref{fig:vcpu_perf}(a), if the network bandwidth requirement is $200$Mbps, {\tt prio2} would need to be assigned since {\tt prio1} achieves only $150$Mbps. However, {\tt prio2} can achieve $306$Mbps, exceeding the bandwidth requirement by $106$Mbps unnecessarily and thus wasting CPU resources.

%\vspace{-0.in}
\begin{figure}
\centering
  \subfloat[Message size: $64$B.] {
    \includegraphics[width=0.22\textwidth]{figure/share_64_fin.pdf}
    \label{fig:vcpu_perf1}
  }
  \subfloat[Message size: $1024$B.] {
    \includegraphics[width=0.22\textwidth]{figure/share_1024_fin.pdf}
    \label{fig:vcpu_perf2}
  }
 % \vspace{-0.1in}
  \caption{vCPU prioritization scheme suffers from inaccurate bandwidth provisioning because of coarse-grained CPU allocations to the Guest. 
  %See \S\ref{subsec:existing} for more details.
  }
  \label{fig:vcpu_perf}
  \vspace{-0.2in}
\end{figure}

The situation becomes even worse with larger message sizes. In Figure~\ref{fig:vcpu_perf}(b), the network bandwidth surges from $250$Mbps to $2.86$Gbps with a message size of $1024$B when the priority is increased from {\tt default} to {\tt prio1}.
This implies that there can be significant over-provisioning of network resources---\eg, if the bandwidth requirement is set slightly above $250$Mbps, which cannot be achieved with the {\tt default} priority, {\tt prio1} would need to be selected, unnecessarily consuming ${\sim}2.61$Gbps of network bandwidth that could have been allocated to other VMs. The results mean imperfect translations due to coarse-grained CPU allocation.

We note that one may consider more precise, finer-grained CPU allocations to enable more accurate rate-limiting. However, this approach involves a trade-off between allocation precision and adjustment time. As demonstrated in \cite{autothrottle23}, CPU allocation can be periodically fine-tuned until the bandwidth requirement is met, but often requires longer adjustment periods. For instance, in the scenario shown in Figure~\ref{fig:vcpu_perf}(a), if the requirement is set to $400$Mbps with the {\tt default} priority, such a fine-tuning scheme could take over $50$ seconds to achieve an SLO-satisfied CPU allocation---by which time most data transfers would likely have already completed~\cite{dctcp}, resulting in a failure to meet the bandwidth requirement.

\vspace{-0.2in}
\subsection{Impact of Different Configurations}
\label{subsec:different}

Now, consider a case where we maintain a long-term record of network bandwidth to CPU mappings for a specific network application. While this approach might enable accurate network bandwidth to CPU translations for that particular application and server configuration, it may still fail to generalize effectively across different workloads or hardware configurations. 
Figure~\ref{fig:comparison} provides evidence of this limitation, showing that bandwidth to CPU translation outcomes can vary significantly across applications and configurations, even when applying the same CPU allocation scheme. 
Specifically, Figure~\ref{fig:comparison}(a) compares the normalized bandwidth of Webserver and Memcached against Netperf, with all applications running on a data size of $64$B using the vCPU prioritization scheme described in \S\ref{subsec:existing}.
Notably, Webserver and Memcached show bandwidth differences ranging from $1.2\times$ to $2.6\times$ relative to Netperf.

%\vspace{-0.2in}
\begin{figure}
\centering
  \subfloat[Different workloads.] {
    \includegraphics[width=0.225\textwidth]{figure/compare_workload_atc.pdf}
    \label{fig:comparison_workloads}
  }
  \subfloat[Different hardware.] {
    \includegraphics[width=0.225\textwidth]{figure/compare_server_atc.pdf}
    \label{fig:comparison_servers}
  }
%  \vspace{-0.1in}
\caption{The outcomes of bandwidth to CPU translation can vary with (a) different workloads under the same hardware configuration ({\tt Config1}) and (b) different hardware configurations but the same application (Netperf). 
\small{(Normalized against (a) Netperf and (b) {\tt Config1}).}
%See \S\ref{subsec:different} for more details.
}
\label{fig:comparison}
\vspace{-0.2in}
\end{figure}

We also test three different hardware configurations listed in Table~\ref{table:configs} using the same application, Netperf.
In Figure~\ref{fig:comparison}(b), the results for {\tt Config2} and {\tt Config3} are normalized against {\tt Config1} (the default configuration), revealing bandwidth variations between $0.4\times$ and $3.5\times$. We observe that the lower memory size in {\tt Config2} limits the increase in network bandwidth, even with higher priority settings (\eg, {\tt prio2} and {\tt prio3} in Figure~\ref{fig:comparison}(b)), resulting in lower bandwidth compared to the default configuration.
These findings indicate that no single universal translation can accommodate all workloads and hardware configurations.

\begin{figure}
  \centering
  \includegraphics[width=0.45\textwidth]{figure/iott_final.pdf}
 % \vspace{-0.1in}
\caption{ \name architecture.
%See \S\ref{subsec:framework} for more details.
}
\label{fig:arch_iott}
\vspace{-0.2in}
\end{figure}


\section{\name Design}
\label{sec:design}

This section presents \name, a general framework designed for translating network bandwidth to CPU resource allocation in virtualized environments. 
\cut{At its core, \name learns the relationship between CPU allocation and network bandwidth for specific workloads. To achieve this, we dynamically collect training data by measuring network bandwidth under varying CPU allocations and use this dataset to train our model. The resulting model can then predict the CPU allocation required to meet the specified network bandwidth.
A key challenge in this process is how to vary CPU allocations for both Host and Guest processes simultaneously during data collection. Figure~\ref{fig:feasibility} reveals that Guest CPU utilization strongly correlates with Host CPU utilization because the Host handles network packet processing for the Guest's applications. In other words, for a specific workload, fixing the Host's CPU allocation effectively determines the Guest's CPU utilization. Leveraging this observation, we simplify the data collection process by varying only the Host CPU allocation while measuring the corresponding Guest CPU utilization as well as network bandwidth. This allows our ML model to predict the required CPU allocations for both Host and Guest for a target network bandwidth, significantly reducing the data collection time.}
In the following subsections, we outline the design of the \name framework (\S\ref{subsec:framework}) and provide an in-depth exploration of our network bandwidth to CPU translation model (\S\ref{subsec:model}). 
We then delve into how \name predicts the appropriate CPU requirement (\S\ref{subsec:optimization}).%Additionally, we delve 

\subsection{Design overview}

\subsection{\name Framework}
\label{subsec:framework}
As a general framework, \name consists of two core components: the \name manager and the \name collector. The \name manager executes the entire workflow, from receiving user-defined network bandwidth requirements to enforcing CPU allocations predicted by our ML model.
The \name collector is responsible for generating the training dataset, derived from measurements of new workloads. And then, the \name manager takes this dataset to train our ML model.
The detailed workflow is outlined as follows (also see Figure~\ref{fig:arch_iott}):




\paragraphb{(1) Training data collection}

First, the \name manager processes user requests containing network bandwidth requirement, target workload, and message size. If a pre-trained ML model for the target workload is unavailable, the \name manager activates the \name collector to gather the required training dataset.
The collector creates a test VM, executes the target workload, and measures the network bandwidth and the Guest CPU utilization while varying Host CPU allocation for the workload. Each measurement runs for $10$ seconds and is repeated three times. 
This process allows \name to learn the CPU usage characteristics for achieving the target bandwidth.
The granularity of Host CPU allocation during data collection will be explained in the following subsections. 
Once the dataset is generated, it is returned to the \name manager.
We note that the \name manager retains all collected datasets for future use. 
Consequently, if a future user request involves the same target workload, the \name manager skips the data collection phase and proceeds directly to subsequent steps.

\paragraphb{(2) Model training and CPU prediction}
Next, the \name manager trains two ML models, Model-G and Model-H, for the Guest and Host, respectively, using the dataset collected by the \name collector. 
These models use Random Forest Regression that shows superior accuracy among representative regression models, as further detailed in \S\ref{subsec:model}.
Model-G predicts the CPU usage of the VM, while Model-H predicts the CPU usage on the Host. The model architecture will be described in the following subsections. 
We note that our model training is notably fast compared to other ML training schemes, which often require several hours to attain the reasonable accuracy of their models~\cite{park2021graf,gan2019seer, gan2021sage}. In contrast, \name can train Model-G and Model-H in only $37$ and $36$ seconds, respectively. %within only ${\sim}30$ seconds each. 
Furthermore, akin to the dataset, the \name manager maintains the trained Model-G and Model-H for workloads, enabling the re-use of pre-trained models for future requests. This significantly improves the efficiency of \name by eliminating redundant processes.
Once Model-G and Model-H are trained, the \name manager uses these models for CPU prediction. 
Here, we adopt a concatenated prediction method, which uses the Guest CPU prediction result as an input to improve the accuracy of Host CPU prediction (further described in \S\ref{subsec:optimization}).


\cut{
\begin{figure*}
\centering
  \subfloat[Linear regression (LR).] {
    \includegraphics[width=0.32\textwidth]{figure/lr_fgcs.pdf}
    \label{fig:lr}
  }
  \subfloat[Support vector regression (SVR).] {
    \includegraphics[width=0.32\textwidth]{figure/svr_fgcs.pdf}
    \label{fig:svr}
  }
  \subfloat[Random forest regression (RFR).] {
    \includegraphics[width=0.32\textwidth]{figure/rfr_fgcs.pdf}
    \label{fig:rfr}
  }
  \caption{While RFR shows the highest predictive accuracy among the three models (\cref{table:model}), it still tends to overestimate or underestimate CPU utilization like the other two models, necessitating further optimization.
  See \S\ref{subsec:model} for more details.}
  \label{fig:model_perf} 
\end{figure*}

\begin{table}
\centering\resizebox{0.8\linewidth}{!}{
\begin{tabular}{@{}crr@{}}
\toprule
  & \multicolumn{1}{c}{RMSLE} & \multicolumn{1}{c}{RMSE} \\ \midrule
 Linear Regression  &  $0.244$                     &  ${11222.513}$                \\
Support Vector Regression &  $0.190$                     &  ${9355.520}$                 \\
Random Forest Regression &  $0.159$                     &  ${8087.562}$                 \\ \bottomrule
\end{tabular}
}
\caption{Random Forest Regression achieves the lowest RMSLE and RMSE among the three models, showcasing the highest predictive accuracy. See \S\ref{subsec:model} for more details.}
\label{table:model}
\end{table}

To further illustrate the evaluation, Figure~\ref{fig:model_perf} compares the actual CPU utilization (x-axis) with the predicted CPU utilization (y-axis)\footnote{In the figure, we present the Host CPU utilization results, as the Guest CPU utilization shows a similar trend.}. Here, the actual utilization refers to the CPU required to achieve the target network bandwidth. In Figure~\ref{fig:model_perf}(a), we observe that predictions from LR consistently overestimate CPU requirements, leading to over-allocation. Over-allocation causes both the Guest and Host to consume more CPU time than needed, resulting in resource inefficiencies. Furthermore, when actual CPU utilization exceeds $75$\%, LR underestimates it, causing under-allocation that leads to SLO violations.
Figure~\ref{fig:model_perf}(b) reveals that SVR yields outcomes similar to those of LR. Although SVR achieves slightly better accuracy in terms of RMSLE and RMSE compared to LR (as shown in \cref{table:model}), it still fails to effectively resolve the allocation inaccuracies.
Lastly, Figure~\ref{fig:model_perf}(c) shows that RFR delivers more accurate predictions, particularly when actual CPU utilization is below $50$\%, outperforming LR and SVR. This is supported by its lowest RMSLE ($0.159$) and RMSE ($8087.562$) values, as shown in \cref{table:model}.
}

\paragraphb{(3) CPU allocation enforcement}
Finally, the \name manager enforces the CPU allocations, predicted by our concatenated method that exploits both Model-G and Model-H, onto the target VM running the target workload.
\name ensures that appropriate CPU resources are allocated to the Host, enabling the SLO-configured VM to achieve optimal CPU utilization for the target bandwidth while minimizing CPU waste.
Specifically, \name uses Linux {\tt cgroup}~\cite{cgroup} to enforce the predicted CPU allocations effectively, rather than modifying Linux CPU scheduling directly.
%A detailed explanation of the CPU allocation enforcement mechanism is provided in \S\ref{sec:implementation}.


\subsection{CPU Translation Model}
\label{subsec:model}
We now introduce the network bandwidth to CPU translation model, which predicts the CPU resources required to meet a specified bandwidth requirement. 
The translation model is developed to accurately capture the relationship between CPU allocation and the resulting network bandwidth.
First, we preprocess the raw data collected by the \name collector, which includes network bandwidth, the number of packets per second (pps), and CPU utilization of the VM (\ie, Guest CPU). These values are collected while varying the Host CPU allocation from $10$\% to $100$\%\footnote{Here, a $100$\% CPU allocation represents the full utilization of a single CPU core.} in increments of $10$\%.
Our preprocessing includes normalization, where min-max scaling adjusts all values to fall between 0 and 1, improving model accuracy. Data is randomly split into training and evaluation sets at a 4:1 ratio. Bandwidth values are converted to Mbps for finer granularity, as Gbps units display only two decimal places.

\kw{Next, we identify an effective regression model for capturing the relationship between CPU allocation and achievable network bandwidth. In virtualized endpoints, this relationship is typically monotone but not strictly linear: bandwidth increases with CPU budget in the low-to-mid regime and saturates near the maximum capacity due to bottlenecks in the host–guest packet-processing pipeline. Therefore, we prioritize models that (i) fit both near-linear trends and saturation effects, (ii) are robust to noise in shared environments, and (iii) incur low training overhead to enable fast retraining. To this end, we evaluate three representative supervised models: Linear Regression (LR), Support Vector Regression (SVR), and Random Forest Regression (RFR). LR serves as a simple baseline for the near-linear regime; SVR provides robustness to noise and outliers; and RFR captures non-linearities and feature interactions arising from message-size effects and hardware configurations. These models are computationally lightweight, allowing \name to retrain quickly as workload specifications change. During evaluation, we tune hyperparameters using random grid search with 5-fold cross-validation, averaging prediction error across folds to obtain a robust validation score.}

\cut{Next, we identify the most effective ML model for capturing the relationship between CPU allocation and network bandwidth. 
To this end, we evaluate three representative supervised ML models for regression: Linear Regression (LR), Support Vector Regression (SVR), and Random Forest Regression (RFR).
During the evaluation of these three models, we perform hyperparameter tuning using the random grid search technique. For each combination of hyperparameters, we apply $5$-fold cross-validation on the training dataset. Specifically, the dataset is divided into five subsets. For each fold, the model is trained on four subsets and validated on the remaining one. The prediction errors across all folds are averaged to calculate the final validation score, providing a robust estimate of the model's performance.}

\renewcommand{\arraystretch}{1.3}
\begin{table}
\footnotesize
\centering
\begin{tabular}{|c|c|c|} \hline
Model   & ~~RMSLE~~  & RMSE  \\ \hline\hline
Linear Regression			    & $0.244$ & ${11222.513}$ \\ \hline
Support Vector Regression   	& $0.190$ & ${9355.520}$ \\ \hline
~~Random Forest Regression~~	& $0.159$ & ${8087.562}$ \\ \hline
\end{tabular}
\vspace{0.1in}
\caption{Random Forest Regression achieves the lowest RMSLE and RMSE among the three models, showcasing the highest predictive accuracy. 
%See \S\ref{subsec:model} for more details.
}
\label{table:model}
\vspace{-0.2in}
\end{table}


In evaluating the prediction error, we use two metrics: Root Mean Square Error (RMSE) and Root Mean Square Log Error (RMSLE). RMSE measures predictive accuracy by summarizing the magnitudes of prediction errors into a single metric, reflecting how closely the data points align with the line of best fit. RMSLE enhances the model's robustness against outliers, ensuring that the model remains effective even when there are significant differences between predicted and actual values, thereby reducing the risk of underestimating key data points.
Table~\ref{table:model} presents the RMSLE and RMSE results for three models trained on the same dataset.
The outcomes reveal that RFR achieves the lowest RMSLE and RMSE values, indicating higher accuracy compared to the other two models.
The reason that RFR outperforms other models is that increasing Host CPU allocation no longer improves network bandwidth once Guest CPU utilization becomes saturated. This nonlinear behavior limits the accuracy of linear models, whereas our RFR effectively captures this nonlinearity, resulting in better prediction performance.
Therefore, we choose RFR as the core model for constructing both Model-G and Model-H.


\subsection{Model Optimization}
\label{subsec:optimization}
This subsection conducts two optimizations for our RFR model: finer-grained data collection and concatenated CPU prediction.

\paragraphb{Finer-grained data collection}
We previously used a $10$\% granularity for varying Host CPU allocation during data collection. To further enhance the model accuracy, finer-grained data collection (\eg, using $1$\% CPU granularity for data collection) could be considered. However, this approach involves a trade-off between collection time and model accuracy. Using finer CPU granularity will increase the data collection time due to the larger number of samples that must be measured. For instance, setting the granularity to $1$\% would increase the data collection time to $50$ minutes, compared to only $4$--$5$ minutes with $10$\% granularity.


To determine an optimal granularity for data collection, we conduct a sensitivity analysis.
Figure \ref{fig:granularity} illustrates how training data granularity impacts the CPU translation accuracy of \name. The results show that as data granularity increases from $1$\% to $10$\%, the range of normalized bandwidth (\ie, measured bandwidth relative to the target bandwidth based on CPU predictions) widens, indicating reduced prediction accuracy. For instance, with a single vCPU, the standard deviation of normalized bandwidth (not shown in the figure) increases from $0.03$ to $0.08$ as granularity increases from $1$\% to $10$\%. However, increasing granularity from $1$\% to $5$\% results in no significant difference, with the maximum observed standard deviation of $0.02$ (for $2$ vCPUs).
Additionally, using $5$\% granularity reduces the data collection time to only $14$ minutes, a significant improvement compared to the $50$ minutes required for $1$\% granularity. Based on these findings, we adopt $5$\% granularity as the default, as it strikes a balance between reduced data collection time and maintaining high accuracy.
Furthermore, we use $1$\% granularity for CPU allocations below $10$\%, as network bandwidth varies more sensitively when CPU utilization is under $10$\%.
Overall, adopting finer granularity (\ie, $1$\% and $5$\%) for data collection improves RMSE by up to $12$\%.


\begin{figure}
  \centering
  \includegraphics[width=0.34\textwidth]{figure/granularity.pdf}
%  \vspace{-0.1in}
\caption{The bandwidth based on the CPU predictions is normalized against the target bandwidth. Finer-grained data collection improves the prediction accuracy.
%See \S\ref{subsec:optimization} for more details.
}
\label{fig:granularity}
\vspace{-0.2in}
\end{figure}

\begin{figure}
  \centering
  \includegraphics[width=0.3\textwidth]{figure/feasibility_tsc.pdf}
%  \vspace{-0.1in}
\caption{Guest CPU utilization is directly affected by Host CPU allocation across different message sizes when running network applications in a virtualized environment. 
%See \S\ref{sec:design} for more details.
}
\label{fig:feasibility}
\vspace{-0.2in}
\end{figure}

\begin{figure*}
\centering
  \subfloat[Separate.] {
    \includegraphics[width=0.13\textwidth]{figure/single_arch_new.pdf}
    \label{fig:single_arch}
  }\hspace{0.2in}
  \subfloat[Concatenated.] {
    \includegraphics[width=0.12\textwidth]{figure/concatenated_arch_new.pdf}
    \label{fig:concatenated_arch}
  }\hspace{0.2in}
  \subfloat[Separate CPU prediction.] {
    \includegraphics[width=0.25\textwidth]{figure/single_new.pdf}
    \label{fig:single_perf}
  }\hspace{0.2in}
  \subfloat[Concatenated CPU prediction.] {
    \includegraphics[width=0.25\textwidth]{figure/concatenated_new.pdf}
    \label{fig:concatenated_perf}
  }  
%  \vspace{-0.1in}
  \caption{{\bf \name model optimization.} Our concatenated prediction method uses the predicted Guest CPU utilization as input for the Host CPU prediction (y-axis in (c) and (d)). 
  %This significantly improves the prediction accuracy compared to the separate method.
  %See \S\ref{subsec:optimization} for more details.
  }
  \label{fig:concatenated}  
  \vspace{-0.2in}
\end{figure*}

\paragraphb{Concatenated CPU prediction}
Figure~\ref{fig:concatenated}(a) illustrates the ``separate'' CPU prediction method (using independent models G and H) previously evaluated in \S\ref{subsec:model}. This method directly predicts the CPU utilization for the Guest and Host based on the network bandwidth requirement and message size. However, it still suffers from over- and under-allocation, as shown in Figure~\ref{fig:concatenated}(c), which compares the target CPU utilization (x-axis) to the predicted Host CPU utilization (y-axis). Here, the target utilization refers to the CPU required to meet the target network bandwidth.


To account for the correlation between the Guest and Host (illustrated in Figure~\ref{fig:feasibility}), we devise a new method called the concatenated CPU prediction method, depicted in Figure~\ref{fig:concatenated}(b). This method predicts the CPU utilization of the Guest and Host in a cascaded manner. First, Model-G takes the bandwidth requirement and message size as inputs and generates a prediction for \textit{Guest CPU} utilization. Subsequently, Model-H receives the predicted \textit{Guest CPU} utilization as input, along with the bandwidth requirement and message size, to predict \textit{Host CPU} utilization.
We then compare the accuracy of the separate and concatenated CPU prediction methods. As shown in Figure~\ref{fig:concatenated}(d), the concatenated prediction method significantly reduces the over- and under-allocation issues observed in the separate prediction method depicted in Figure~\ref{fig:concatenated}(c). For instance, the number of instances where allocation errors exceed $5$\% decreases from $16$ to $8$, representing $50$\% reduction. Furthermore, the concatenated prediction method improves the accuracy, especially for high CPU utilization, by providing predictions that align more closely with the actual values. When evaluated using RMSLE, the concatenated prediction method achieves a value of $0.2068$, which is approximately $7.1$\% lower than the $0.2214$ RMSLE achieved by the separate prediction method\footnote{Note that RMSLE increases in Figure~\ref{fig:concatenated} compared to Table~\ref{table:model} because the number of data samples increases by double with the finer-grained data collection.}.

\cut{
\section{\name Implementation}
\label{sec:implementation}

We implement the \name manager and the collector as user-level daemons in Linux.
The workflow begins with the \name manager receiving HTTP POST requests from users in {\tt json} format. These requests specify the target bandwidth value (\ie, network throughput), the target workload, and message size. \cut{Upon receiving a request, the \name manager checks whether pre-trained Model-G and Model-H are available for the target workload. 
If the pre-trained models are available, the \name manager directly performs CPU predictions without activating the \name collector. Otherwise, the \name manager triggers the \name collector to generate a new training dataset for the target workload.}
Note that when \name manager receives user requests, the target workload runs under the default CPU scheduler until the trained models for CPU translation are ready.

For training data collection, the \name collector issues a set of commands via {\tt ssh} to a test VM to execute the target workload.
While the test VM runs the workload, the \name collector measures key metrics such as network bandwidth, CPU utilization, and the number of packets processed per second.
These measurements are then compiled into a {\tt csv} file and sent to the \name manager.
The entire data collection process takes approximately $14$ minutes in our default hardware configuration (see {\tt Config1} in Table~\ref{table:configs} and \S\ref{subsec:evalsetup}).
This is significantly faster than other approaches that rely on historical usage data~\cite{bashir2021take} or reinforcement learning~\cite{qiu2020firm}, often requiring over an hour to complete. Considering that recent data analytics and enterprise workloads typically run for several hours~\cite{rzadca2020autopilot, bashir2021take, amvrosiadis2018diversity, cano2016characterizing}, the data collection time required by \name is practical for deployment in such long-running scenarios.

The \name collector provides a CPU contention-free environment to the test VM during data collection, allowing the CPU allocation for the corresponding Host process to vary up to $100$\%.
However, in real-world scenarios where multiple VMs share the same physical cores, CPU contention may prevent SLO-configured VMs from achieving the required CPU utilization.
To address this, \name first evenly distributes vCPUs across physical cores to minimize CPU contention among SLO-configured VMs. And then it leverages the Linux {\tt cgroup} mechanism~\cite{cgroup} to allow the SLO-configured VM to consume more CPU time than its fair share, ensuring that the target VM receives sufficient CPU resources to meet its bandwidth requirements.% 

For model training, the \name manager preprocesses the data received from the \name collector, adapting it to the model's feature requirements and converting it into tensors. These tensors are then used to train both Model-G and Model-H.
Note that both Model-G and Model-H are based on the default parameter setting of {\tt RandomForestRegressor} in {\tt scikit-learn}, which includes {\tt n\_estimators}=100 (the number of decision trees) and {\tt max\_depth}=None (allowing each tree to expand fully until it reaches pure leaf nodes or until other stopping criteria are met). The choice of {\tt n\_estimators}=100 balances the prediction accuracy with computational efficiency, while setting {\tt max\_depth}=None enables the model to capture complex patterns in the data.
Finally, the \name manager enforces the Host CPU allocation from our concatenated prediction method. 
Specifically, the \name manager configures the CPU quota for the Host. For instance, if the predicted \textit{Host CPU} allocation is $15$\%, the \name manager sets {\tt cpu.cfs\_quota\_us} to $15000$ while keeping the default value of {\tt cpu.cfs\_period\_us} at $100000$, thereby allowing the Host to use up to $15$\% of the total CPU capacity.
}
\section{Evaluation}
\label{sec:eval}
The \name framework is implemented as two user-level daemons in Linux: the \name manager and the \name collector. The manager receives user requests in JSON format via HTTP POST, specifying the target bandwidth, workload, and message size, and initially executes the workload under the default CPU scheduler until the trained models for CPU translation become available. For model training, the collector remotely executes the target workload on a test VM via {\tt ssh}, gathers key performance metrics such as bandwidth, CPU utilization, and packet processing rate, and returns them as a {\tt csv} dataset to the manager. The collected data are then preprocessed and transformed into tensors for training two Random Forest models (Model-G and Model-H) using the default {\tt scikit-learn} parameters ({\tt n\_estimators}=100, {\tt max\_depth}=None), balancing accuracy and computational cost. Finally, the \name manager enforces the predicted Host CPU allocation by configuring the Linux {\tt cpu.cfs\_quota\_us} parameter to reflect the model’s output.

This section evaluates \name in a KVM-based virtualized environment.
We begin by outlining our evaluation setup (\S\ref{subsec:evalsetup}) and then present the results in the following subsections.
First, we evaluate \name performance by comparing it to existing network scheduling and vCPU prioritization schemes (\S\ref{subsec:evaltc}).
Next, we provide a qualitative comparison of \name with recent studies that leverage state-of-the-art ML techniques, including FIRM~\cite{qiu2020firm}, Sinan~\cite{zhang2021sinan}, and Autothrottle~\cite{wang2024autothrottle} (\S\ref{subsec:evalcomp}), followed by a performance comparison with a widely used deep learning model (\S\ref{subsec:evalmlp}).
We then demonstrate the effectiveness of \name with real-world applications such as Webserver and Memcached (\S\ref{subsec:evaldiff}).
Finally, we describe the overheads of \name on unseen workloads (\S\ref{apdx:evalpred}).
Several additional evaluation results (including performance with noisy neighbors, performance with different hardware configurations, etc.) can be found in Appendix~\ref{apdx}.


\subsection{Evaluation Setup}
\label{subsec:evalsetup}

As described in \S\ref{sec:design}, \name offers a framework capable of building a translation model for new hardware setups within $16$ minutes, while consistently maintaining accurate prediction performance across different hardware configurations.
Consequently, our evaluation focuses on the default hardware setup, {\tt Config1} in Table~\ref{table:configs} (See further results for {\tt Config2} in Section~\ref{subsec:evaltc}, which show a similar performance trend to {\tt Config1}).%~\ref{apdx:evalconf}).

\paragraphb{Hardware setup}
We use two physical machines equipped with an Intel Xeon E5-2650v2@2.6GHz 10-core processor with hyper-threading off, $256$GB memory, and a $512$GB SSD. Also, the machines have a dual-port Intel 82599EB $10$Gbps NIC that is connected to each other via a $10$Gbps switch.
Both machines run Linux kernel 5.4 and enable virtualization using KVM. We then create an SLO-configured VM that operates with the same Linux kernel version.
To maximize CPU efficiency in network processing~\cite{cai2021}, we enable TCP Segmentation Offload (TSO) and Generic Receive Offload (GRO) at both guest and host. The maximum transmission unit (MTU) size is $1500$B.

\paragraphb{Evaluated workloads}
We primarily use Netperf for our microbenchmark; our Netperf workload continuously generates TCP messages (ranging from $64$B to $1024$B) and sends them to the peer machine, which makes it suitable for the baseline evaluation. 
We also evaluate \name with two real-world applications, Apache Webserver, and Memcached.
To generate HTTP requests, we use the {\tt wrk} benchmarking tool, which retrieves $1$KB files from the Webserver. Note that the file size is chosen based on typical log sizes found in production datacenters~\cite{sizeofthelogfile}.
In Memcached workloads, we employ the Memaslap tool~\cite{memaslap} to send requests to the Memcached server, consisting of a distribution of $10$\% SET and $90$\% GET requests. We use the default settings of Memaslap with $64$B keys and $1$KB values~\cite{han2012megapipe}.
Lastly, we use two performance metrics: (1) normalized bandwidth, the network bandwidth normalized to the target network bandwidth, and (2) total CPU utilization, including both Host and Guest CPU usage.


\subsection{Microbenchmarks}
\label{subsec:evaltc}

We first evaluate \name in terms of CPU prediction accuracy and CPU efficiency with Netperf benchmarks, 
using the same configurations described in \S\ref{sec:motivation}.

\begin{figure}
\centering
  \subfloat[Message size: $64$B.] {
    \includegraphics[width=0.225\textwidth]{figure/compare_tc64_fin.pdf}
    \label{fig:netperf_64_share}
  }
  \subfloat[Message size: $1024$B.] {
    \includegraphics[width=0.225\textwidth]{figure/compare_tc1024_fin.pdf}
    \label{fig:netper_1024_share}
  }
%  \vspace{-0.1in}
  \caption{\name (TS) reduces the total CPU utilization compared to {\tt tc} while meeting the bandwidth requirements accurately.
  %See \S\ref{subsec:evaltc} for more details.
  }
  \label{fig:comparison_tc}
  \vspace{-0.2in}
\end{figure}

\begin{figure}
\centering
  \subfloat[Message size: $64$B.] {
    \includegraphics[width=0.225\textwidth]{figure/compare_share64_fin.pdf}
    \label{fig:comparison_share1}
  }
  \subfloat[Message size: $1024$B.] {
    \includegraphics[width=0.225\textwidth]{figure/compare_share1024_pact.pdf}
    \label{fig:comparison_share2}
  }
  \caption{\name (TS) performs accurate CPU predictions in all cases, while vCPU prioritization (PR) suffers from high fluctuation in network bandwidth.}
  \label{fig:comparison_share} 
  \vspace{-0.2in}
\end{figure}

\paragraphb{Comparison with network scheduling scheme}
Similar to Figure~\ref{fig:network_perf}, we increase the target network bandwidth from $100$Mbps to $400$Mbps with message sizes of $64$B and $1024$B.
In Figure~\ref{fig:comparison_tc}, the normalized bandwidth decreases as the target bandwidth increases when using {\tt tc} for both $64$B and $1024$B messages. This result indicates that {\tt tc} struggles to achieve accurate rate-limiting due to insufficient CPU resources. 
For instance, {\tt tc} achieves the normalized bandwidth of $0.98$ for $64$B messages with the $100$Mbps requirement, when ample CPU resources are available. 
However, as the target bandwidth increases to $400$Mbps, the normalized bandwidth drops to $0.38$ (approximately $\frac{150}{400}$) as shown in Figure~\ref{fig:comparison_tc}(a), because {\tt tc} reaches up to $150$Mbps with $100$\% CPU utilization, as discussed in Figure~\ref{fig:network_perf}(a).
On the other hand, in Figure~\ref{fig:comparison_tc}, \name consistently achieves precise rate-limiting accuracy with the normalized bandwidth of $1.00$ regardless of the message size. 
Especially, we observe that \name achieves this level of accuracy while maintaining significantly lower CPU utilization, offering ${\sim}6.5\times$ improvements over {\tt tc} in terms of total CPU utilization. 
Our further investigation reveals that providing lower CPU allocation to the Host can positively impact the overall network processing capacity. 
It allows more data to accumulate in the Host's Tx queue before the CPU is scheduled for Host processing, increasing the likelihood of generating larger packets via TSO. As a result, per-byte network processing overhead can be significantly reduced. For instance, when the target bandwidth is set to $300$Mbps with a message size of $1024$B (Figure~\ref{fig:comparison_tc}(b)), {\tt tc} achieves an average packet size of only $1.1$KB, whereas \name increases the average packet size to $62.4$KB.
This highlights \name's ability to optimize CPU allocation accurately and effectively.


\definecolor{ForestGreen}{RGB}{34,139,34}
\renewcommand{\arraystretch}{1.3}
\begin{table*}
\centering\resizebox{\linewidth}{!}{
\begin{tabular}{|c|c|c|c|c|c|}
\hline
\textbf{Title}        & \textbf{Model}                & \textbf{Input}                                 & \textbf{Output}      & \textbf{Data collection time} & \textbf{Training time}                                                \\ \hline\hline
\textbf{FIRM}~\cite{qiu2020firm}         & SVM+Reinforcement learning    & Resource usage, performance counts             & Resource limits      & \textcolor{black}{1 hour}                          & \textcolor{black}{1 hour}                                                            \\ \hline
\textbf{Sinan}~\cite{zhang2021sinan}        & CNN+Boosted trees             & Resource usage \& allocation, latency         & Tail latency         & \textcolor{red}{15 hours}                      & \textcolor{ForestGreen}{33 seconds} \\ \hline %\begin{tabular}[c]{@{}c@{}}\textcolor{ForestGreen}{33 seconds}\\ (on NVIDIA Titan XP)\end{tabular} \\ \hline
\textbf{Autothrottle}~\cite{wang2024autothrottle}& Online reinforcement learning & RPS, tail latencies, actual CPU allocation & CPU throttle targets & \textcolor{red}{12 hours}                        & \textcolor{ForestGreen}{$>$60 seconds}                                                               \\ \hline\hline
\textbf{\name}               & Random forest regression      & Target bandwidth, message size, pps                         & CPU quota            & \textcolor{ForestGreen}{14 minutes}                        & \textcolor{ForestGreen}{73 seconds} \\ \hline %\begin{tabular}[c]{@{}c@{}}\textcolor{ForestGreen}{11 seconds}\\ (on Xeon CPU)\end{tabular}        
\end{tabular}}
  \vspace{0.1in}
  \caption{\textbf{Comparison between \name and recent ML schemes.} \name minimizes data collection and model training time by utilizing relatively simple ML techniques, such as random forest regression, compared to previous studies. 
  %See \S\ref{subsec:evalcomp} for more details.
  }
  \label{tab:comparison}
 \vspace{-0.2in}
\end{table*}
\paragraphb{Comparison with vCPU prioritization}
In Figure~\ref{fig:comparison_share}, we observe that vCPU prioritization suffers from high fluctuation in the normalized bandwidth in most cases, as discussed in \S\ref{sec:motivation}. For instance, with a message size of $64$B, the normalized bandwidth reaches ${\sim}1.53$ (Figure~\ref{fig:comparison_share}(a)). This increases further to ${\sim}7.15$ with a message size of $1024$B (Figure~\ref{fig:comparison_share}(b)) due to reduced per-byte packet processing overhead with larger messages.
In contrast, \name effectively allocates appropriate CPU budgets to both the Guest and Host based on its CPU prediction model. This enables \name to consistently meet network bandwidth requirements, regardless of message size, achieving an average normalized bandwidth of $1.01$. 
We note that the vCPU prioritization scheme always shows $100$\% core utilization, as it performs work-conserving scheduling using Linux CFS. In comparison, \name and {\tt tc} are non-work-conserving by regulating CPU quota or the volume of transmitted packets.


\begin{comment}
\paragraphb{Performance with Different Hardware}\label{apdx:evalconf}
We repeat the experiment using {\tt Config2} (as detailed in Table~\ref{table:configs}) to see whether \name's effectiveness holds across different configurations. The {\tt Config2} servers are connected via $10$Gbps links, with all other software configurations remaining consistent with \S\ref{sec:eval}.

Figures~\ref{fig:config2_result}(a) and (b) demonstrate \name's performance compared to {\tt tc}; similar to Figures~\ref{fig:comparison_tc}, \name achieves accurate bandwidth requirements while significantly reducing total CPU utilization. In Figures~\ref{fig:config2_result}(c) and (d), \name consistently provides accurate CPU predictions, while vCPU prioritization continues to show high fluctuations in network bandwidth, aligning with the results observed in Figure~\ref{fig:comparison_share}.
\end{comment}




\subsection{Comparison with Recent ML Schemes}
\label{subsec:evalcomp}

Recent studies have developed various ML models to achieve target SLOs for containers, leveraging state-of-the-art ML techniques such as deep learning and reinforcement learning (RL).
For instance, Sinan~\cite{zhang2021sinan} employs a convolutional neural network (CNN) and boosted trees to assist clusters in allocating appropriate computing resources to meet latency SLOs. Similarly, FIRM~\cite{qiu2020firm} and Microsoft Autothrottle~\cite{wang2024autothrottle} seek to determine the optimal allocation of computing resources, such as CPU time, memory bandwidth, and last-level cache (LLC) bandwidth, using RL models. Table~\ref{tab:comparison} provides a comparison between \name and recent relevant studies, focusing on the model details, input and output values, and the time required for data collection and model training. 
It indicates that CNN- and RL-based approaches often require longer data collection or training periods to achieve high accuracy. 
For example, Sinan~\cite{zhang2021sinan} and Autothrottle~\cite{wang2024autothrottle} require $15$ and $12$ hours, respectively, to collect training data.  FIRM~\cite{qiu2020firm} takes a modest time of $1$ hour for both data collection~\cite{firm-code} and training.
\name further reduces these times, requiring only $14$ minutes for data collection and $73$ seconds for model-G and model-H training by employing a relatively simple model---Random Forest Regression---while effectively predicting CPU allocations for both the Guest and Host.


We also aimed to directly compare \name with these techniques in terms of model accuracy %and training efficiency 
to demonstrate the effectiveness of our \name model.
However, using their existing implementations~\cite{sinan-code,firm-code,autothrottle-code} was challenging for two reasons. First, they are specifically optimized for microservices where each service runs in individual containers, and SLOs are evaluated across all associated containers in an end-to-end manner. Consequently, their models are designed to control resource allocation for bottleneck microservices across containers, whereas \name manages CPU resources at the VM level. Second, their models incorporate additional features such as memory/disk usage, latency, and LLC metrics, while \name primarily focuses on CPU allocation. 
These differences result in unfair comparisons under our evaluation scenarios.
In fact, applying our Netperf dataset directly to their implementations yields poor performance. Therefore, instead of making direct comparisons, we develop a Multi-Layer Perceptron (MLP) model optimized for our Netperf dataset to evaluate \name's effectiveness against recent ML models, as discussed in the following subsection.


\begin{figure}
\centering
  \subfloat[1 vCPU.] {
    \includegraphics[width=0.225\textwidth]{figure/MLP_1vcpu_atc.pdf}
    \label{fig:mlp1}
  }
  \subfloat[8 vCPUs.] {
    \includegraphics[width=0.225\textwidth]{figure/MLP_8vcpu_atc.pdf}
    \label{fig:mlp2}
  }
  \caption{{\bf MLP performance.} 
  The MLP-based prediction shows $1.3\times$ lower accuracy (RMSE) compared to \name, even after $2000$ epochs. %despite a large number of epochs, requiring $61$\% more training time compared to \name. See \S\ref{subsec:evalmlp} for more details.
  }
  \label{fig:mlp}
   \vspace{-0.2in}
\end{figure}


\subsection{Comparison with MLP}
\label{subsec:evalmlp}
We implement an MLP model, a widely used deep-learning approach for solving a variety of supervised learning tasks, including regression. MLP offers a good balance between accuracy and computational efficiency. Our MLP model consists of six linear layers and uses Mean Square Error (MSE) as the loss function. The number of layers is empirically determined by evaluating RMSE to ensure optimal performance.
For the activation function, we use the Gaussian Error Linear Unit (GELU), which applies the Gaussian cumulative distribution function to assign weights to inputs based on their values. GELU provides smoother and more efficient nonlinear transformations for our dataset compared to other activation functions.

We further optimize the model using the AdamW optimizer with a learning rate of $0.001$. Hyperparameters, such as the number of epochs ($2000$) and batch size ($256$), are tuned via grid search optimization. The model is trained on Netperf data, with message sizes ranging from $64$B to $1024$B. Key training features include network throughput, Host and Guest CPU utilization, the number of packets processed per second, and the Host CPU quota ratio. 
The data is split into training and test sets with an $80$--$20$ ratio, and features are normalized to a range of $0$ to $1$ using {\tt MinMaxScaler} to ensure balanced weighting during training. To monitor performance and prevent overfitting, we perform periodic validation every $400$ epochs.

Figure~\ref{fig:mlp} illustrates the model accuracy in terms of RMSE, normalized against \name's performance, and the training time for our MLP model. For a fair comparison, we use the same dataset, which consists of $360$ samples.
As shown in Figure~\ref{fig:mlp}, the accuracy of the MLP model improves with an increasing number of epochs, but \name still achieves better RMSE performance---$1.4\times$ better with $1$ vCPU (Figure~\ref{fig:mlp}(a)) and $1.3\times$ better with $8$ vCPUs (Figure~\ref{fig:mlp}(b)) at $2000$ epochs. 
At this point, the training time for the MLP model increases to $38$ seconds for both $1$- and $8$-vCPU cases. While \name requires a longer training time ($73$ seconds, as shown in Table~\ref{tab:comparison}), achieving $1.3$--$1.4\times$ better accuracy for an additional $35$ seconds of training time is a reasonable tradeoff.

\begin{figure}
  \centering
  \includegraphics[width=0.38\textwidth]{figure/macro_mlp.pdf}
%  \vspace{-0.1in}
  \caption{
  {\bf \name is effective across different workloads}, by more accurately meeting the given bandwidth requirements (\ie, achieving a normalized bandwidth close to $1.0$) compared to {\tt tc} and the MLP model. \small{The `+' symbol represents the mean value.}
 % See \S\ref{subsec:evaldiff} for more details.
 }
  \label{fig:diffwork}
  \vspace{-0.2in}
\end{figure}
\subsection{Real-world Applications}
\label{subsec:evaldiff}

\cut{We further evaluate the effectiveness of \name with two real-world applications, Apache Webserver and Memcached, while allocating $8$ vCPUs to the target VM. First, we test a fixed-load scenario by maintaining constant traffic loads for the applications to demonstrate how \name meets the target bandwidth across different workloads. Next, we explore a dynamic-load scenario by varying the traffic load below the target bandwidth to show that \name's ability to handle dynamic workloads effectively.}
\kw{We evaluate \name using two widely deployed applications: Apache Webserver and Memcached. Unless stated otherwise, the target VM is configured with 8 vCPUs.}

\paragraphb{Webserver traffic model}
\kw{We generate HTTP traffic using a \emph{closed-loop concurrency} model: a fixed number of clients maintain concurrent TCP connections and issue a new request immediately after completing the previous one (i.e., no think time). 
\textbf{Fixed-load:} we sustain 1{,}000 concurrent connections repeatedly fetching a 1\,KB static object to emulate a saturated small-object/API-like workload.
\textbf{Dynamic-load:} we vary concurrency from 512 to 4 and 2 to emulate demand drop and near-idle periods.
This setup produces small request/response messages; packetization is governed by TCP MSS and HTTP headers, yielding high packet-processing pressure at the host under high concurrency.}

%\paragraphb{Memcached traffic model}
%\kw{We emulate a read-heavy caching tier with 1{,}000 concurrent clients issuing 100{,}000 operations per trial using the default GET/SET ratio (9:1). This workload generates high-rate, small request/response messages and stresses the host-side vhost/softirq processing path, allowing us to evaluate bandwidth fulfillment under realistic cache-style access patterns.}



\paragraphb{Fixed-load scenario}
For each application, we measure network bandwidth over $100$ iterations, randomly selecting target bandwidth values between $100$Mbps and its maximum achievable throughput.
The normalized bandwidth, calculated relative to the target bandwidth, is plotted in Figure~\ref{fig:diffwork}, comparing \name with {\tt tc} and our MLP model (\S\ref{subsec:evalmlp}).
In the figure, \name consistently outperforms the other schemes, achieving an average normalized bandwidth of $1.0$ for Webserver and $0.99$ for Memcached, accurately meeting the bandwidth requirements.
To further evaluate the effectiveness of \name, we define ``(normalized) bandwidth variation'' as the standard deviation of the normalized target bandwidth ($1.0$). 
\name achieves $2.25\times$ lower bandwidth variation than both {\tt tc} and the MLP model for Webserver, and $3.0\times$ and $2.2\times$ lower variation than {\tt tc} and the MLP model, respectively, for Memcached.

We observe that {\tt tc} struggles to meet high target bandwidth values exceeding $800$Mbps for Webserver and $200$Mbps for Memcached. These results align with Figure~\ref{fig:comparison_share}, showing the limitations of network scheduling-based solutions. The MLP model shows either lower accuracy in achieving target bandwidth (with Webserver) or higher bandwidth variation (with Memcached) compared to \name, due to its lower CPU prediction accuracy as discussed in \S\ref{subsec:evalmlp}.

\begin{figure}
% \vspace{-0.4in}
\centering
  \subfloat[Normalized throughput.] {
    \includegraphics[width=0.22\textwidth]{figure/dynamic_web_atc.pdf}
    \label{fig:dynamic1}
  }
  \subfloat[CPU utilization(\%).] {
    \includegraphics[width=0.225\textwidth]{figure/dynamic_web_cpu_fin.pdf}
    \label{fig:dynamic2}
  }
%  \vspace{-0.1in}
  \caption{\name performance under dynamic traffic loads (Webserver on VM1).}
  \label{fig:dynamic_web} 
\end{figure}

\begin{table}[htbp]
\centering
\small % Slightly smaller but still standard readable size
\setlength{\tabcolsep}{3pt} % Tighten horizontal space
\caption{\kw{Over-provisioning waste in CPU utilization}}
\begin{tabular}{@{}lccccccc@{}}
\toprule
\textbf{Interval (s)} & \textbf{0} & \textbf{30} & \textbf{60} & \textbf{90} & \textbf{120} & \textbf{150} & \textbf{180} \\ \midrule
Avg. Load (\%)        & 100           & 25             & 12.5           & 0               & 12.5             & 25               & 100              \\
Avg. Waste (\%)       & 2.5           & 2.5            & 18.0           & 56.1            & 17.8             & 2.6              & 3.0              \\
VM2 Change (\%)       & --            & -10.7          & 82.0           & 121.3           & -120.6           & -70.2            & -17.3            \\ \bottomrule
\end{tabular}
\end{table}
\begin{comment}
\paragraphb{Dynamic-load scenario} In this experiment, we run two VMs: VM1, running Webserver with a target bandwidth of $1$Gbps and VM2, running Sysbench, a CPU-intensive workload managed by the default CPU scheduler instead of \name. Both VMs are configured with $8$ vCPUs, sharing the same physical CPU cores. 
Initially, \name allocates the predicted CPU resources to VM1, allowing it to achieve a normalized throughput of $1.0$, meeting the target bandwidth. After $30$ seconds, the traffic load of Webserver is gradually reduced to $0$ and then increased back to $1$Gbps, as shown in Figure~\ref{fig:dynamic_web}(a).
We also plot the Sysbench throughput, normalized to its maximum throughput at $800$\% CPU utilization. When Webserver generates no traffic load (between $90$--$120$ seconds), Sysbench achieves a normalized throughput of $1.0$, fully utilizing the available CPU resources. 
Figure~\ref{fig:dynamic_web}(b) illustrates how \name dynamically reallocates unused CPU utilization from Webserver to Sysbench during traffic variations. 
This is possible because configuring CPU quota in \name allows the work-conserving CPU scheduler to redistribute unused CPU resources.
A similar trend is observed with Memcached (see Appendix~\ref{apdx:evalmem}).
\end{comment}

\paragraphb{Dynamic-load scenario}
\kw{In this experiment, we run two VMs: VM1 running Webserver with a target bandwidth of $1$Gbps and VM2 running Sysbench, a CPU-intensive workload managed by the default CPU scheduler instead of \name. Both VMs are configured with $8$ vCPUs and share the same physical CPU cores.
Initially, \name applies the predicted host CPU quota for VM1, enabling VM1 to meet the target bandwidth (normalized throughput $=1.0$). After $30$ seconds, the Webserver load is gradually reduced to $0$ and then increased back to $1$Gbps (Figure~\ref{fig:dynamic_web}(a)).
When VM1 becomes idle (between $90$--$120$ seconds), Sysbench reaches its maximum normalized throughput ($=1.0$), indicating that the CPU budget reserved as VM1's quota is not consumed and is reclaimed by the work-conserving scheduler.}

\kw{Importantly, \name enforces a \emph{maximum} CPU budget on the host-side packet-processing path; under off-peak load, VM1's packet-processing threads are not continuously runnable, so the enforced quota does not translate into persistent CPU occupation.
To quantify the resulting trade-off, we additionally report the gap between \name's allocated CPU cap and the CPU actually used (and the minimum CPU required to sustain the instantaneous load), showing that while worst-case provisioning sets a conservative cap, the scheduler reclaims the unused portion and makes it available to other VMs during off-peak periods (Figure~\ref{fig:dynamic_web}(c)).
A similar trend is observed with Memcached (Appendix~\ref{apdx:evalmem}).}





\cut{
\subsection{Performance with Noisy neighbors}
\label{apdx:noisy}

We introduce noisy neighbors to evaluate how \name performs under high contention for host resources; noisy neighbors~\cite{pu2012your} aggressively consume computing resources as the target VM and bring resource contention that can affect the network performance of the target VM. For this experiment, we set a specific network requirement ($200$Mbps) with $1024$B messages -- we choose this configuration as the normalized bandwidth is nearly $1.0$ for all schemes (see Figure~\ref{fig:comparison_tc}(b) and Figure~\ref{fig:comparison_share}(b)).
Then, we use two metrics: SLO violation rate and bandwidth variation (defined in \S\ref{subsec:evaldiff}) to represent the effectiveness of the translation in meeting the bandwidth requirement (SLO). The SLO violation rate identifies cases where the normalized bandwidth falls below $0.95$, indicating instances where the bandwidth requirement is not met.

We run three types of noisy neighbors each of which significantly consumes either CPU, last-level cache (LLC), or network bandwidth.
First, the CPU-bound neighbor is the customized CPU stressor based on iBench and {\tt stree-ng}~\cite{qiu2020firm}. It exhausts all available CPU cores through intensive floating-point calculations, integer operations, and bit manipulation. Second, the LLC neighbor utilizes iBench and {\tt pmbw} to perform both streaming and random access operations to fully utilize the bandwidth and capacity of the entire LLC~\cite{qiu2020firm}.
Lastly, the network-bound neighbor executes Netperf with default configuration settings, including $16$KB message sizes and no bandwidth limitation to fully utilize the network capacity of the host server.
\begin{table}
\centering
\resizebox{0.9\linewidth}{!}{
\begin{tabular}{@{}crrr@{}}
\toprule
\multicolumn{1}{l}{} & \multicolumn{1}{c}{Tasador} & \multicolumn{1}{c}{Traffic control} & \multicolumn{1}{c}{Prioritization} \\ \midrule
CPU                  & 0\% (0.2)                     & 20\% (7.6)                          & 5\% (25.5)                         \\
L3                   & 0\% (0.0)                       & 0\% (6.1)                             & 5\% (47.7)                         \\
Network              & 0\% (0.1)                     & 10\% (7.3)                          & 0\% (72.9)                           \\ \bottomrule
\end{tabular}}
  \vspace{0.1in}
  \caption{\name is rarely affected by noisy neighbors in terms of both SLO violation rate (\%) and bandwidth variation (in parentheses).
  %See \S\ref{apdx:noisy} for more details.
  }
  \label{fig:anomaly}
  \vspace{-0.2in}
\end{table}

Table~\ref{fig:anomaly} shows that \name remains unaffected by the presence of noisy neighbors -- \ie, SLO violations are consistently at zero with \name, regardless of the type of noisy neighbors. Also, \name maintains minimal bandwidth variation (parenthesized in Table~\ref{fig:anomaly}), which means that the target VM consistently attains network performance close to the target bandwidth.
On the other hand, {\tt tc} increases the SLO violation rate because VMs running alongside noisy neighbors may struggle to meet their bandwidth requirements. In particular, when the target VM runs with a CPU-bound neighbor, {\tt tc} experiences high SLO violation rate of $20$\%. This is primarily due to the insufficient CPU allocation the target VM receives as a result of contention with the CPU-bound neighbor. In comparison, vCPU prioritization reduces SLO violations compared to {\tt tc}. However, it shows high bandwidth variation due to inaccurate CPU allocation. This implies that the target VM receives significantly higher network bandwidth than the specified SLO. With vCPU prioritization, the bandwidth variation increases, ranging from $25.5$ to $72.9$, resulting in network bandwidth much higher than the target bandwidth. For example, the target VM achieves an average network bandwidth of $277$Mbps when running with a network-bound neighbor, which is $39$\% higher than the target bandwidth.}


\subsection{\name Overheads for Unseen Workloads}
\label{apdx:evalpred}

When \name receives requests for unseen workloads beyond Netperf, Memcached, and Webserver, \name follows three steps, as illustrated in Figure~\ref{fig:arch_iott}. The overall overheads introduced by unseen workloads are categorized into data collection overhead (steps $2$--$4$ in Figure~\ref{fig:arch_iott}), training overhead (step 5), and prediction overhead (step 6).
We measure the data collection and training overheads as depicted in Table 3; it is observed that TASADOR takes around 14 minutes and 73 seconds, respectively, regardless of the unseen workloads.
This result shows that \name's data collection and training time remain consistent even when workloads show different runtime behavior. This is due to the design of \name; the Host CPU allocation is done independently of workload behavior. Specifically, we increase the Host CPU allocation from 1\% to 100\%, generating 81 measurement samples regardless of workloads. The measurement duration for each sample is also fixed, which leads to consistent data collection and training time across all workloads.

In this subsection, we evaluate the prediction overhead, which may increase as the number of user requests to the \name manager increases. To quantify this overhead, we measure the prediction delay---the time required to complete CPU prediction in the \name manager---as the number of concurrent user requests ($N$) increases, as shown in Figure~\ref{fig:latency}.
Our results show that the average latency increases from $135$ms to $387$ms as the number of requests increases from $100$ to $10000$, and more than $75$\% of user requests are handled within $300$ms when $N{<}1000$.
Even under the worst case, the maximum delay remains below $0.8$s with $10000$ concurrent requests. These results suggest that \name can efficiently handle many concurrent requests within $1$ second, which is sufficiently tolerant for real-world cluster environments where workloads typically run for extended durations~\cite{roy2015inside, dhukic2019advance}.

\begin{figure}
  \centering
  \includegraphics[width=0.38\textwidth]{figure/latency_pact25.pdf}
  \vspace{-0.1in}
  \caption{The prediction latency of \name is relatively low (0.8s at maximum) even though the number of concurrent requests increases to 10000. 
  %See \S\ref{apdx:evalpred} for more details.
  }
  \label{fig:latency}
  \vspace{-0.2in}
\end{figure}




\section{Discussion}
Based on our design and evaluation, this section further explores potential extensions for \name.

\paragraphb{Why a simple Random Forest model is effective for bandwidth-to-CPU translation}
\kw{Although prior ML-based resource managers learn policies over multiple resources and objectives (e.g., latency under multi-dimensional contention), the bandwidth-to-CPU translation targeted by \name exhibits a structured relationship that is well-suited to lightweight regression.
For a fixed workload and traffic specification, achieved bandwidth typically increases with CPU budget but \emph{saturates} near the maximum capacity due to bottlenecks in the end-host packet-processing pipeline (e.g., softirq/vhost/virtio and application processing).
This produces a monotone yet non-linear curve with a near-linear region at low CPU budgets and diminishing returns as the system approaches its throughput ceiling.
Random Forest Regression (RFR) matches this structure: as an ensemble of decision trees, it captures saturation and interaction effects (e.g., message size/packet rate and hardware configuration) without extensive feature engineering, while remaining fast to train from modest datasets.}

\kw{We also considered more complex learning approaches such as Multi-Layer Perceptrons (MLP) and reinforcement learning (RL).
While MLPs are expressive, they often require larger datasets and careful tuning to generalize reliably under a tight profiling budget, which can reduce robustness in practice.
RL is attractive for closed-loop control under highly dynamic and partially observed environments; however, for immediate bandwidth-to-CPU mapping it typically requires online exploration or iterative adjustment, which can introduce transient under-provisioning or oscillations during convergence.
Such behavior is undesirable when bandwidth requirements must be met consistently, especially under bursty traffic or contention~\cite{delimitrou2014quasar,gao2018rlnet}.
By exploiting the structured bandwidth--CPU relationship and training within $\sim$16 minutes per workload specification, \name achieves a favorable accuracy--overhead trade-off: it provides stable, one-shot predictions at deployment time without exploration-driven fluctuations.
As future work, \name could be combined with lightweight online calibration or safe adaptation mechanisms to further improve robustness under long-term workload drift.}


\paragraphb{Support for container-based microservice architectures}
\kw{In container-based environments, microservices introduce additional challenges due to their distributed nature: multiple containers collectively form a single service, each container may have distinct resource demands, and containers may be placed across different VMs.
Previous studies primarily focus on container-level resource allocation to satisfy latency objectives~\cite{zhang2021sinan,qiu2020firm,wang2024autothrottle}, whereas \name targets VM-level CPU provisioning to satisfy network bandwidth requirements.
These approaches are complementary in modern clouds where microservices often run atop VMs: \name can ensure that the underlying VM provides sufficient CPU capacity for bandwidth delivery, while container-level controllers can further allocate CPU among microservices to meet latency requirements.
A promising direction is a unified framework that combines VM-level bandwidth-to-CPU translation with container-level latency-oriented allocation, enabling consistent performance isolation and end-to-end QoS for complex microservice deployments.}



\paragraphb{Support for different virtualization architectures}
Although this paper focuses on a representative software-based virtualization solution (KVM), the \name framework is also applicable to other architectures, such as paravirtualization solutions like Xen~\cite{barham2003xen} and hardware-assisted solutions like Single Root I/O Virtualization (SR-IOV). In Xen, Dom0 (the control domain) handles network processing for guest VMs and employs dedicated network threads for individual VMs. Therefore, \name’s Model-H can be directly applied to allocate CPU resources for these Dom0 network threads, ensuring accurate and efficient resource provisioning.
With SR-IOV, \name’s architecture becomes even simpler because VMs have direct access to virtual network functions through hardware. This eliminates the need for Model-H, allowing \name to train only Model-G and simplifying the optimization process by avoiding the model concatenation step described in \S\ref{subsec:optimization}.

%\paragraphb{Support for container-based microservice architectures}
%In container-based environments, microservices introduce additional challenges due to their distributed nature. In a microservice architecture, multiple containers collectively form a single service. Each container may have different resource requirements and could be distributed across VMs. Previous studies focus on container-level resource allocation to meet latency SLOs~\cite{zhang2021sinan,qiu2020firm,wang2024autothrottle}, while our work addresses VM-level allocation for network bandwidth requirements. Thus, a promising future direction would be to integrate these approaches into a unified framework to more effectively support diverse cloud workloads.

\paragraphb{Resource allocation for different computing resources}
In our evaluation, we assume that VMs have sufficient memory (\eg, $4$GB) to achieve target network bandwidth based on the CPU allocation provided by the \name manager. 
However, VM workload performance can degrade if the VM has insufficient memory~\cite{chen2019parties}. We believe that the \name manager can be extended to incorporate other computing resources, such as memory, as additional input features for Model-G and Model-H. For example, when the \name collector executes the target workload on a test VM to collect training data, it can also measure the VM's memory usage using tools like {\tt vmstat}. The \name manager can then train Model-G and Model-H with this extended dataset, including memory usage as a feature. Subsequently, the memory allocation predicted by the models can be enforced using the Linux {\tt cgroup} subsystem, ensuring that the target VM has proper memory to achieve its configured requirements.

\paragraphb{Handling diverse network traffic patterns}
This paper evaluates \name with three different workloads. Here, we would like to discuss how \name can handle workloads with other traffic patterns. For example, when a workload has a range of message sizes, \name selects parameters representing the worst case in terms of CPU demand for the workload. Specifically, \name can use the smallest possible message size, as smaller messages generate more packets and incur higher network processing overhead~\cite{han2012megapipe}. Similarly, regarding traffic patterns, \name can adopt the most bursty pattern that yields peak load during data collection. Thus, \name can adapt itself for the worst case to handle a range of workloads.

However, we find that this behavior of \name often results in over-provisioning of CPU, particularly under non-peak load conditions. Over-provisioning can waste valuable CPU resources. But note that \name is designed to exploit the work-conserving property of the Linux CFS scheduler. When a workload does not fully utilize the allocated CPU, \name lets CFS automatically redistribute the unused CPU cycles to other VMs. As demonstrated in Figure~\ref{fig:dynamic_web}, \name performs well under such dynamic-load scenarios with varying traffic loads.
Also, in our evaluation, the VMs are tested under a single stream of network traffic. It is because we want \name to be evaluated without other interferences. Nevertheless, \name is capable of handling multiple traffic streams. When there are multiple streams of requests, \name can allocate additional Host processes for the multiple streams. Since network traffic in KVM is processed by the Host, \name can map each incoming stream to a distinct Host process. This mapping enables CPU allocation to be managed per stream.

\newcommand{\lt}{<}
\paragraphb{Overhead for data collection}
\name utilizes two test VMs for data collection: one as the sender and the other as the receiver for a workload.
A test VM is allocated with a Host process, and the Host process handles the network processing via \texttt{virtio} driver, as described in Section~\ref{sec:motivation}. Thus, the overhead of training data collection comes from two test VMs, including two Host processes and their resource usage.
Our experiments show that, for the Netperf workload with a single vCPU assigned, the sender test VM consumes up to 180\% of CPU resources. Note that the receiver VM utilizes much less CPU(\lt $100$\%). \kw{Mention FLASH HERE}
%This measurement captures the maximum observed overhead attributable to the data collection process.

\section{Related Work}
\vspace{-0.05in}
%Previous studies on achieving network bandwidth requirements in virtualized environments can be categorized into network scheduling, vCPU prioritization, fine-grained CPU allocation, and ML approaches.
%\noindent\textbf{Summary of differences.}
\kw{Prior work either (i) shapes network egress without translating bandwidth intent into CPU budgets,
(ii) biases scheduling toward I/O-oriented vCPUs without computing the CPU needed for a specific bandwidth target,
(iii) reallocates whole cores or iteratively tunes quotas for latency objectives, or
(iv) applies ML for multi-resource SLO control mainly in native/container settings.
In contrast, \name targets \emph{VM-level bandwidth requirements} and introduces a lightweight, host--guest-aware model that directly predicts the host CPU quota needed to sustain a given bandwidth target.}


\paragraphb{Network scheduling}~\cite{jeyakumar2013eyeq,jang2015silo, dmvl, elasticswitch} proposes to control the network bandwidth of VMs in hypervisor by offering an independent packet queue to each VM and managing the transmission rate of the packet queue. This enables VMs to achieve different network bandwidths independent of other VMs.
However, VMs can fail to achieve network bandwidth without the right CPU budgets because studies on network scheduling assume that CPU allocation is always sufficient.
As network scheduling techniques only allocate network bandwidth and do not properly intervene in CPU allocation, VMs cannot meet bandwidth requirements when CPU becomes the bottleneck.

\paragraphb{vCPU prioritization}~\cite{xu2013vturbo,suo2017preserving,cheng2012vbalance,ahn2018accelerating,jia2020vsmt,jia2018effectively} suggests improving vCPU scheduling to improve the network performance of VMs. They point out that the root cause of the low network bandwidth is scheduling latency due to CPU sharing between VMs on a physical machine. In particular, when VMs run different types of workloads, such as CPU-bound and I/O-bound, the VMs running I/O-bound workloads experience significant scheduling delays and suffer from performance degradation.
To resolve the scheduling latency issue, several studies~\cite{xu2013vturbo,ahn2018accelerating} propose to decrease the scheduling period in a hypervisor.
In addition, other studies~\cite{suo2017preserving,jia2020vsmt,jia2018effectively} suggest preferentially scheduling the vCPUs of VMs running I/O-bound workloads.
%Although studies on vCPU prioritization are mostly effective in improving performance, they have difficulty achieving network bandwidth in a fine-grained manner, as we observe in \S\ref{subsec:existing}.
\kw{These techniques improve scheduling responsiveness for I/O-oriented VMs, but they do not compute the CPU budget required to sustain a given bandwidth target under diverse traffic patterns and contention levels.
\name instead provides a direct bandwidth-to-CPU translation and enforces the predicted host CPU quota, enabling fine-grained bandwidth provisioning beyond weight-based prioritization.}

\paragraphb{Fine-grained CPU allocation}
Optimizing system resource allocation is a well-known challenge~\cite{bird2011pacora,colmenares2013tessellation}. 
Recent studies have explored fine-grained CPU allocation to meet user-specified SLOs. For instance, Shenango~\cite{shenango} and Caladan~\cite{fried2020caladan} achieve target latency by reallocating CPUs to latency-sensitive applications at microsecond granularity.
However, these systems allocate entire CPU cores rather than fractional CPU utilization to enable fast CPU reallocations. This often allocates more CPU time than is required for a specific bandwidth requirement.
In~\cite{autothrottle23}, the authors propose a mechanism that iteratively adjusts CPU quotas until SLO-configured applications meet their bandwidth requirements. Although this approach enables fine-grained CPU allocations, it often requires numerous iterations to identify the appropriate CPU quota, resulting in prolonged allocation times.

\paragraphb{Machine learning approaches}
Recently, many studies have adopted ML techniques to develop innovative solutions for performance monitoring~\cite{gan2019seer,gan2021sage}, task scheduling~\cite{zhang2021sinan,mao2016resource}, and resource allocation~\cite{qiu2020firm, qiu2023aware,qiu2022simppo,wang2022deepscaling}. For instance, FIRM~\cite{qiu2020firm} introduces a low-level controller that partitions computing resources by utilizing Linux functionalities, such as {\tt tc}, and predicts the required resources to meet SLOs using ML and DL models. While these studies, demonstrate the effectiveness of their proposed techniques in mitigating SLO violations, they may not be directly applicable to the virtualized environments focused on in this paper. This limitation arises because these studies primarily address native environments or container-based systems, without considering the unique CPU usage dynamics of both the guest and host in virtualized settings, \kw{where packet processing spans guest vCPUs and host-side vhost/qemu/softirq paths.}


\vspace{-0.05in}
\section{Conclusion}

This paper demonstrates that translating "network bandwidth to CPU" is not straightforward in virtualized environments. The exact amount of CPU allocation needed for a specific bandwidth requirement is dependent on many factors, including workloads, message sizes, and hardware configurations.
We present \name, an ML framework for the network bandwidth to CPU translation. \name performs three main tasks: (1) training data collection, (2) model training and CPU prediction, and (3) CPU allocation enforcement. We implement \name in a KVM-based virtualized environment. \name requires only ${\sim}16$ minutes for collecting data and training models, which makes it easily adaptable to new workloads. Furthermore, \name achieves ${\sim}6.5\times$ improvements over the existing schemes regarding CPU allocation efficiency. We believe that \name is generalizable, as it collects training data for individual workloads and builds tailored models specific to each scenario. \kw{As future work, we plan to extend \name to include a closed-loop monitoring and feedback system. By periodically comparing real-time bandwidth performance against predicted CPU usage, the framework will be able to handle rare mispredictions or environmental drifts. This extension will allow dynamic re-allocation, further optimizing resource oversubscription and ensuring strict SLO adherence even in highly volatile cloud environments.}


\cut{This paper demonstrates that translating ``network bandwidth to CPU'' is not straightforward in virtualized environments. The exact amount of CPU allocation needed for the bandwidth requirement is dependent on many factors, including workloads, message sizes, and hardware configurations.
We present \name, an ML framework for the network bandwidth to CPU translation. \name performs three main tasks: (1) training data collection, (2) model training and CPU prediction, and (3) CPU allocation enforcement. We implement \name in a KVM-based virtualized environment. \name requires only ${\sim}16$ minutes for collecting data and training models, which makes it easily adaptable to new workloads. 
Furthermore, \name achieves ${\sim}6.5\times$ improvements over the existing schemes regarding CPU allocation efficiency. We believe that \name is generalizable, as it collects training data for individual workloads and builds tailored Model-G and Model-H specific to each workload.}

%\section*{Acknowledgement}
%This work was supported by National Research Foundation of Korea funded by the Ministry of Science, ICT (No. NRF-2019H1D8A2105513). 

%\section*{Acknowledgment}

% Can use something like this to put references on a page
% by themselves when using endfloat and the captionsoff option.
\ifCLASSOPTIONcaptionsoff
  \newpage
\fi



% trigger a \newpage just before the given reference
% number - used to balance the columns on the last page
% adjust value as needed - may need to be readjusted if
% the document is modified later
%\IEEEtriggeratref{8}
% The "triggered" command can be changed if desired:
%\IEEEtriggercmd{\enlargethispage{-5in}}

% references section

% can use a bibliography generated by BibTeX as a .bbl file
% BibTeX documentation can be easily obtained at:
% http://mirror.ctan.org/biblio/bibtex/contrib/doc/
% The IEEEtran BibTeX style support page is at:
% http://www.michaelshell.org/tex/ieeetran/bibtex/
%\bibliographystyle{IEEEtran}
% argument is your BibTeX string definitions and bibliography database(s)
%\bibliography{IEEEabrv,../bib/paper}
%
% <OR> manually copy in the resultant .bbl file
% set second argument of \begin to the number of references
% (used to reserve space for the reference number labels box)

\bibliographystyle{IEEEtran}
%\bibliographystyle{IEEEtran-firstonly}
\bibliography{paper}
% biography section
% 
% If you have an EPS/PDF photo (graphicx package needed) extra braces are
% needed around the contents of the optional argument to biography to prevent
% the LaTeX parser from getting confused when it sees the complicated
% \includegraphics command within an optional argument. (You could create
% your own custom macro containing the \includegraphics command to make things
% simpler here.)
%\begin{IEEEbiography}[{\includegraphics[width=1in,height=1.25in,clip,keepaspectratio]{mshell}}]{Michael Shell}
% or if you just want to reserve a space for a photo:

\begin{comment}
\begin{IEEEbiography}[{\includegraphics[width=1in,height=1.25in,clip,keepaspectratio]{figure/kwlee}}]{Kyungwoon Lee}
received the B.E. degree from the School of Electronics Engineering,
Kyungpook National University, Daegu, South Korea, and the M.S. and Ph.D. degrees in computer science from Korea University, Seoul, South Korea. From 2020 to 2022, she was with the Department of Computer science and Engineering as a research professor. She is currently working as an assistant professor in the School of Electronics Engineering, Kyungpook National University. Her research interests include resource scheduling in cloud computing, container and server virtualization, and TCP/IP kernel networking stack.
\end{IEEEbiography}%
\begin{IEEEbiography}[{\includegraphics[width=1in,height=1.25in,clip,keepaspectratio]{figure/우수연.pdf}}]{Sooyeon Woo}

received the B.E. degree from the School of Electronics Engineering,
Kyungpook National University, Daegu, South Korea. She is currently pursuing an M.S. degree in electronics engineering from the same university. Her research interests include cloud computing and deep-learning techniques.
\end{IEEEbiography}%
\begin{IEEEbiography}[{\includegraphics[width=1in,height=1.25in,clip,keepaspectratio]{figure/jaehyun}}]{Jaehyun Hwang}
received the B.S. degree from The Catholic University of Korea, South Korea, in 2003, and the M.S. and Ph.D. degrees from Korea University, Seoul, South Korea, in 2005 and 2010, respectively, all in computer science. He was with Bell Labs, Alcatel-Lucent, Seoul, South Korea, as a Member of Technical Staff from 2010 to 2015 and Samsung Electronics, Suwon, South Korea, as a Senior Engineer from 2015 to 2018. He was also with the Department of Computer Science, Cornell University, Ithaca, NY, USA, as a Postdoctoral Associate from 2018 to 2021. Since 2022, he has been with the Department of Semiconductor Systems Engineering, Sungkyunkwan University, Suwon, South Korea, as an assistant professor.
His current research interests include storage
disaggregation, CPU-efficient datacenter systems, and cloud computing.
\end{IEEEbiography}%
\begin{IEEEbiography}[{\includegraphics[width=1in,height=1.25in,clip,keepaspectratio]{figure/chuckyoo}}]{Chuck Yoo [M]}
received the B.S. and M.S. degrees in electronic engineering from Seoul National University, and M.S. and Ph.D. degrees in computer science from the University of Michigan, Ann Arbor. He worked as a researcher at Sun Microsystems. Since 1995, he has been at the College of Informatics at Korea University, where he is currently a professor. His research interests include server/network virtualization and operating systems.
\end{IEEEbiography}
\end{comment}
%\vfill

% Can be used to pull up biographies so that the bottom of the last one
% is flush with the other column.
%\enlargethispage{-5in}

\clearpage
\begin{figure}
\centering
  \subfloat[Message size: $64$B.] {
    \includegraphics[width=0.22\textwidth]{figure/612_share_64.pdf}
    \label{fig:vcpu_612_64}
  }
  \subfloat[Message size: $1024$B.] {
    \includegraphics[width=0.22\textwidth]{figure/612_share_1024.pdf}
    \label{fig:vcpu_612_1024}
  }
 % \vspace{-0.1in}
  \caption{Changes in network bandwidth depending on the vCPU priority level differs in newer Linux kernel.
  }
  \label{fig:vcpu_612}
  \vspace{-0.2in}
\end{figure}



\begin{figure}[!ht]
\centering
  \subfloat[Message size: $64$B.] {
    \includegraphics[width=0.22\textwidth]{figure/config2_64.pdf}
    \label{fig:config2_64}
  }
  \subfloat[Message size: $1024$B.] {
    \includegraphics[width=0.22\textwidth]{figure/config2_1024.pdf}
    \label{fig:config2_1024}
  }\\
  \subfloat[Message size: $64$B.] {
    \includegraphics[width=0.22\textwidth]{figure/config2_share_64_new.pdf}
    \label{fig:config2_64_share}
  }
  \subfloat[Message size: $1024$B.] {
    \includegraphics[width=0.22\textwidth]{figure/config2_share_1024.pdf}
    \label{fig:config2_1024_share}
  }\caption{{\bf \name performance with {\tt Config2}.}
  (a)(b) Compared to {\tt tc}, \name (TS) reduces the total CPU utilization while meeting the bandwidth requirements accurately. (c)(d) Compared to vCPU prioritization, \name (TS) performs accurate CPU predictions in all cases while vCPU prioritization (PR) suffers from high fluctuation in network bandwidth, consistent with the results in Figures~\ref{fig:comparison_tc} and~\ref{fig:comparison_share}.
  %See \S\ref{apdx:evalconf} for more details.
  }
  \label{fig:config2_result}
\end{figure}
\appendix
\section{Appendix}
\label{apdx}
This section provides additional evaluation results, including performance with different hardware configurations, noisy neighbors, and a dynamic-load scenario with Memcached.

\subsection{Performance with Different Hardware}
\label{apdx:evalconf}

We repeat the microbenchmark evaluation from \S\ref{subsec:evaltc} using {\tt Config2} (as detailed in Table~\ref{table:configs}) to see whether \name's effectiveness holds across different configurations. The {\tt Config2} servers are connected via $10$Gbps links, with all other software configurations remaining consistent with \S\ref{sec:eval}.

Figures~\ref{fig:config2_result}(a) and (b) demonstrate \name's performance compared to {\tt tc}; similar to Figures~\ref{fig:comparison_tc}, \name achieves accurate bandwidth requirements while significantly reducing total CPU utilization. In Figures~\ref{fig:config2_result}(c) and (d), \name consistently provides accurate CPU predictions, while vCPU prioritization continues to show high fluctuations in network bandwidth, aligning with the results observed in Figure~\ref{fig:comparison_share}.

\subsection{Performance with Noisy neighbors}
\label{apdx:noisy}
\begin{table}
\centering
\resizebox{0.9\linewidth}{!}{
\begin{tabular}{@{}crrr@{}}
\toprule
\multicolumn{1}{l}{} & \multicolumn{1}{c}{Tasador} & \multicolumn{1}{c}{Traffic control} & \multicolumn{1}{c}{Prioritization} \\ \midrule
CPU                  & 0\% (0.2)                     & 20\% (7.6)                          & 5\% (25.5)                         \\
L3                   & 0\% (0.0)                       & 0\% (6.1)                             & 5\% (47.7)                         \\
Network              & 0\% (0.1)                     & 10\% (7.3)                          & 0\% (72.9)                           \\ \bottomrule
\end{tabular}}
  \vspace{0.1in}
  \caption{\name is rarely affected by noisy neighbors in terms of both SLO violation rate (\%) and bandwidth variation (in parentheses).
  }
  \label{fig:anomaly}
  %\vspace{-0.2in}
\end{table}

\kw{To evaluate \name's robustness under high host-resource contention, we introduce "noisy neighbors"~\cite{pu2012your} that aggressively consume resources shared with the target VM. We set a target network requirement of $200$ Mbps with $1024$ B messages, a configuration where the normalized bandwidth is approximately $1.0$ across all schemes under ideal conditions (see Figure~\ref{fig:comparison_tc}(b)). Then, we employ two metrics to evaluate SLO adherence: the SLO violation rate (the number of instances where normalized bandwidth drops below $0.95$) and bandwidth variation. We test against three types of stressors:
\begin{itemize}
\item \textbf{CPU-bound:} Exhausts all cores using floating-point and integer operations via \texttt{stress-ng}.
\item \textbf{LLC-bound:} High-frequency random and streaming memory access via \texttt{pmbw} to saturate the last-level cache.
\item \textbf{Network-bound:} Saturated host network capacity using \texttt{Netperf} with $16$ KB messages and no rate limiting.\end{itemize}}

\kw{As shown in Table~\ref{fig:anomaly}, \name remains virtually unaffected by contention, maintaining a $0$\% SLO violation rate and negligible bandwidth variation. This resilience stems from \name's direct mapping and isolation mechanism. Unlike Linux \texttt{tc}, which operates primarily at the networking layer and relies on the underlying OS scheduler for CPU time, \name proactively translates the $200$ Mbps requirement into the exact CPU cycles necessary to sustain that throughput. By explicitly allocating these calculated CPU shares to the target VM's vCPU, \name ensures that the resources required for network processing are strictly reserved and isolated from the scheduler's attempts to balance load with the CPU-bound neighbor.}

\kw{In contrast, \texttt{tc} suffers a $20$\% violation rate under CPU contention. Because \texttt{tc} lacks a mechanism to guarantee the CPU cycles required to process packets at the specified rate, the target VM's network processing is frequently preempted or delayed by the CPU-bound stressor. While vCPU prioritization reduces these violations, it lacks the precision of \name's translation model, leading to significant over-provisioning and high bandwidth variation (up to $72.9$). For instance, under network-bound noise, prioritization results in a $39$\% higher bandwidth than requested ($277$ Mbps).}

\kw{Ultimately, \name's advantage lies in its ability to treat network SLOs as a multi-resource allocation problem. By shielding the network-related CPU demand from host-level contention, \name maintains stable performance where traditional packet-level scheduling (\texttt{tc}) or coarse-grained prioritization fails.}



\cut{
We introduce noisy neighbors to evaluate how \name performs under high contention for host resources;
noisy neighbors~\cite{pu2012your} aggressively consume computing resources as the target VM and bring resource contention that can affect the network performance of the target VM. For this experiment, we set a specific network requirement ($200$Mbps) with $1024$B messages -- we choose this configuration as the normalized bandwidth is nearly $1.0$ for all schemes (see Figure~\ref{fig:comparison_tc}(b) and Figure~\ref{fig:comparison_share}(b)).
Then, we use two metrics: SLO violation rate and bandwidth variation (defined in \S\ref{subsec:evaldiff}) to represent the effectiveness of the translation in meeting the bandwidth requirement (SLO). The SLO violation rate identifies cases where the normalized bandwidth falls below $0.95$, indicating instances where the bandwidth requirement is not met.


We run three types of noisy neighbors, each of which significantly consumes either CPU, last-level cache (LLC), or network bandwidth.
First, the CPU-bound neighbor is the customized CPU stressor based on iBench and {\tt stree-ng}~\cite{qiu2020firm}. It exhausts all available CPU cores through intensive floating-point calculations, integer operations, and bit manipulation. Second, the LLC neighbor utilizes iBench and {\tt pmbw} to perform both streaming and random access operations to fully utilize the bandwidth and capacity of the entire LLC~\cite{qiu2020firm}.
Lastly, the network-bound neighbor executes Netperf with default configuration settings, including $16$KB message sizes and no bandwidth limitation to fully utilize the network capacity of the host server.

Table~\ref{fig:anomaly} shows that \name remains unaffected by the presence of noisy neighbors -- \ie, SLO violations are consistently at zero with \name, regardless of the type of noisy neighbors. Also, \name maintains minimal bandwidth variation (parenthesized in Table~\ref{fig:anomaly}), which means that the target VM consistently attains network performance close to the target bandwidth.
On the other hand, {\tt tc} increases the SLO violation rate because VMs running alongside noisy neighbors may struggle to meet their bandwidth requirements. In particular, when the target VM runs with a CPU-bound neighbor, {\tt tc} experiences high SLO violation rate of $20$\%. This is primarily due to the insufficient CPU allocation the target VM receives as a result of contention with the CPU-bound neighbor. In comparison, vCPU prioritization reduces SLO violations compared to {\tt tc}. However, it shows high bandwidth variation due to inaccurate CPU allocation. This implies that the target VM receives significantly higher network bandwidth than the specified SLO. With vCPU prioritization, the bandwidth variation increases, ranging from $25.5$ to $72.9$, resulting in network bandwidth much higher than the target bandwidth. For example, the target VM achieves an average network bandwidth of $277$Mbps when running with a network-bound neighbor, which is $39$\% higher than the target bandwidth.}




\subsection{Dynamic-load Scenario with Memcached}
\label{apdx:evalmem}
\begin{figure}
\centering
  \subfloat[Normalized throughput.] {
    \includegraphics[width=0.22\textwidth]{figure/dynamic_mem_atc.pdf}
    \label{fig:dynamic3}
  }
  \subfloat[CPU utilization(\%).] {
    \includegraphics[width=0.225\textwidth]{figure/dynamic_mem_cpu_test.pdf}
    \label{fig:dynamic4}
  }
  \caption{{\bf \name performance under dynamic traffic loads (Memcached on VM1).} \name allows the CPU-intensive VM2 (Sysbench) to dynamically utilize the unused CPU resources as VM1's traffic load varies.
  VM1's CPU utilization includes the Host's CPU usage.}
  \label{fig:dynamic_mem} 
\end{figure}

We repeat the dynamic-load scenario from \S\ref{subsec:evaldiff} with Memcached, as shown in Figure~\ref{fig:dynamic_mem}. 
Similar to Figure~\ref{fig:dynamic_web}, we run two VMs: VM1, running Memcached with a target bandwidth of $1.5$Gbps, and VM2, running Sysbench, a CPU-intensive workload managed by the Linux CPU scheduler rather than \name. Both VMs are configured with $8$ vCPUs, sharing the same physical CPU cores. 
The results show a similar trend to Figure~\ref{fig:dynamic_web}, demonstrating that \name dynamically reallocates unused CPU utilization from Memcached to Sysbench during traffic variations (Figure~\ref{fig:dynamic_mem}(b)).






% that's all folks
\end{document}


